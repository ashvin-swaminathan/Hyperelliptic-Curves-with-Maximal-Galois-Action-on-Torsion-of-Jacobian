\section{Introduction}
\subsection{Background}
\label{subsection:intro-background}
Let $A$ be a principally polarized abelian variety (PPAV) of dimension $g \geq 1$
over a number field $K$.
Fix an algebraic closure $\ol{K}$ of $K$, and let $G_K \defeq \Gal(\ol{K}/K)$ be the absolute Galois group. 
The action of $G_K$ on the torsion points of $A(\ol{K})$ gives rise to the 
\emph{adelic} Galois representation
$$\rho_A \colon G_K \to\GSp_{2g}(\wh{\ZZ}).$$
For prime numbers $\ell$, the \emph{mod-$\ell$ Galois representation} $\rho_{A,\ell} \colon G_K \to \GSp_{2g}(\bz/\ell\bz)$ is defined by reducing the image of $\rho_A$ modulo $\ell$.
See~\cite[Section 2.2]{seaweed} and
\cite[Section 3.1]{landesman-swaminathan-tao-xu:rational-families} for more detailed descriptions of these representations.

In 1972, Jean-Pierre Serre proved the celebrated Open Image Theorem (see~\cite{causalrelationship}), which states that for an elliptic curve $E/K$ without complex multiplication, $\rho_E(G_K)$ is an open subgroup of, and hence has finite index in, the profinite group $\GSp_{2}(\wh{\ZZ})$. 
While the Open Image Theorem implies that the adelic Galois representation maps onto a large subgroup of $\GSp_{2g}(\wh{\ZZ})$, the image of this representation is not always equal to $\GSp_{2g}(\wh{\ZZ})$. 
Indeed, Serre observed in~\cite[Proposition 22]{causalrelationship} that for every elliptic curve $E/\QQ$, the image of $\rho_E$ has even index in $\GSp_2(\wh{\ZZ})$.
Nonetheless, in~\cite[Sections 5.5.6-8]{causalrelationship},
Serre constructs several examples of elliptic curves over $\bq$
whose Galois representations have ``maximal image'' among all elliptic curves, in the sense that the index of the \mbox{image in $\GSp_2(\widehat{\mathbb Z})$ is equal to $2$.}

The obstruction faced by elliptic curves over $\QQ$ to having surjective adelic Galois representation no longer exists when $\QQ$ is replaced by a larger number field. 
In~\cite{greasy}, Greicius constructs an example of an elliptic curve over a cubic extension of $\bq$ whose Galois representation
has image equal to $\GSp_2(\widehat{\mathbb Z})$.
Furthermore, in~\cite{seaweed}, Zywina constructs an example of a non-hyperelliptic curve of genus $3$ over $\bq$ whose Jacobian has adelic Galois image equal to $\GSp_6(\widehat{\mathbb Z})$. 
While there are explicit examples in genera $1$ and $3$, to the authors' knowledge,
there are no examples in the literature of curves of genus 
$2$ with associated Galois representation having maximal image among such curves.
Additionally, there are no known examples of hyperelliptic curves of genus $3$ whose Galois image is maximal.
Nevertheless, there are a few examples that come close:
	In~\cite[Theorem 5.4]{dooleyfat},
	Dieulefait gives an example of a
	genus-$2$ curve over $\mathbb Q$
	whose Jacobian has mod-$\ell$
	monodromy equal to $\GSp_4(\mathbb Z/\ell \mathbb Z)$
	for $\ell \ge 5$.
	Similarly, in~\cite[Corollary 1.1]{anni2016residual},
	an example of a genus-$3$ hyperelliptic curve
	over $\bq$ whose Jacobian has mod-$\ell$ Galois image equal to $\GSp_6(\mathbb Z/\ell \mathbb Z)$
	for primes $\ell \geq 3$ is constructed.
	However, in both of these cases, it is easy to check 
	that these examples have mod-2
	Galois image that is not maximal
	among all hyperelliptic curves of genus
	$2$ or $3$.
	In Theorem~\ref{exemplinongratia},
	we improve on the results of~\cite{dooleyfat} and~\cite{anni2016residual},
	giving explicit examples of hyperelliptic
	curves of genus $2$ and $3$ over $\QQ$ with maximal adelic Galois image.
	The reader may wish to also refer to the related recent paper~\cite{anni2017constructing}, which constructs
	hyperelliptic curves with maximal mod-$\ell$ Galois image in all genera $g$ with the property that $2g+2$ can be expressed as of sum of two primes in two different ways, with none of the primes being the largest prime less than $2g + 2$.\footnote{Note that~\cite{anni2017constructing}
	therefore does not address the cases $g = 2, 3$,
	which we cover in this paper.}

In addition to finding explicit examples of PPAVs with maximal Galois image, there are a number of results in the literature concerning how many members of a given family of PPAVs have maximal adelic Galois image.
The first key result in this direction is due to Duke, who proved in~\cite{duke:elliptic-curves-with-no-exceptional-primes} that ``most'' elliptic curves $E/\QQ$ in the standard family with Weierstrass equation $y^2 = x^3 + ax + b$ have the property that $\rho_{E,\ell}(G_{\QQ}) = \GSp_2(\ZZ/\ell \ZZ)$ for every prime number $\ell$; here, the term ``most'' means a density-$1$ subset of curves ordered by na\"{i}ve height. Building upon the work of Duke, Jones proved in~\cite[Theorem 4]{josofabank} that $[\GSp_{2g}(\wh{\ZZ}) : \rho_E(G_K)] = 2$ for most elliptic curves $E$ in the standard family over $\QQ$.
In~\cite[Theorem 1.15]{zywina2010hilbert}, Zywina generalized 
the above results, showing that most members of every non-isotrivial rational family of elliptic curves over an arbitrary number field have maximal adelic Galois image, subject to the constraints that arise from the arithmetic and geometric properties of the family under consideration. 
Additional results over $\mathbb Q$ were
obtained in~\cite{grant:a-formula-for-the-number-of-elliptic-curves-with-exceptional-primes},~\cite{cojocaruH:uniform-results-for-serres-theorem-for-elliptic-curves}, and~\cite{cojocaruGJ:one-parameter-families-of-elliptic-curves}
(see~\cite[p.~6]{zywina2010hilbert} for a more detailed overview).
In Theorem~\ref{mainbldg}, we give an explicit version of~\cite[Theorem 1.1]{landesman-swaminathan-tao-xu:rational-families} -- a result that generalizes Zywina's results to rational families of higher-dimensional PPAVs --
for many common families of hyperelliptic curves.
This yields a generalization of \cite[Theorem 1.2]{zywina2010elliptic}
and~\cite[Theorem 4]{josofabank} to hyperelliptic curves of higher genus.

\subsection{Main Results}

In this paper, we primarily consider those PPAVs that arise as Jacobians of hyperelliptic curves belonging to one of the following four standard families; we restrict our consideration to curves of genus at least $2$ because the results of Zywina in~\cite{zywina2010hilbert} completely handle the case of elliptic curves.
\begin{definition}
	\label{definition:standard-families}
Let $g \geq 2$ be an integer, and for $i \in \{1, 2, 3, 4 \}$ define $\standardTarget g i K$ by
\begin{align*}
&\standardTarget g 1 K = \mathbb A^{2g+1}_{[a_0, \ldots, a_{2g}]} \setminus \Delta^{(1)}, \,\,\,\,\, \standardTarget g 2 K = \mathbb A^{2g+2}_{[a_0, \ldots, a_{2g+1}]}\setminus \Delta^{(2)}, \\
&\,\,\,\,\,  \standardTarget g 3 K = \mathbb A^{2g}_{[a_0, \ldots, a_{2g-1}]}\setminus \Delta^{(3)}, \,\,\,\,\,  \standardTarget g 4 K = \mathbb A^{2g+1}_{[a_0, \ldots, a_{2g}]} \setminus \Delta^{(4)}, 
\end{align*}
where each $\Delta^{(i)}$ is the discriminant locus, on which the indicated polynomial has at least one multiple root: 
\[
\begin{array}{cc}
x^{2g+1} + a_{2g}x^{2g} + \cdots + a_0  &\rightsquigarrow  \Delta^{(1)}  \\
x^{2g+2} + a_{2g+1}x^{2g+1} + \cdots + a_0 & \rightsquigarrow  \Delta^{(2)}\\
x^{2g+1} + a_{2g-1}x^{2g-1} + \cdots + a_0  &\rightsquigarrow \Delta^{(3)}\\
x^{2g+2} + a_{2g}x^{2g} + \cdots + a_0 &\rightsquigarrow \Delta^{(4)}.
\end{array}
\]
Consider the following vanishing loci, and view them as families over $\standardTarget g i K$ via projection onto the first factor:
\begin{align*}
V(y^2 - x^{2g+1} - a_{2g}x^{2g} - \cdots - a_0) &\hookrightarrow \standardTarget g 1 K \times \mathbb A^2_{[x,y]} \to \standardTarget g 1 K, \\
V(y^2 - x^{2g+2} - a_{2g+1}x^{2g+1} - \cdots - a_0) &\hookrightarrow \standardTarget g 2 K \times \mathbb A^2_{[x,y]} \to \standardTarget g 2 K,  \\
V(y^2 - x^{2g+1} - a_{2g-1}x^{2g-1} - \cdots - a_0) &\hookrightarrow \standardTarget g 3 K\times \mathbb A^2_{[x,y]} \to  \standardTarget g 3 K, \\
V(y^2 - x^{2g+2} - a_{2g}x^{2g} - \cdots - a_0) &\hookrightarrow \standardTarget g 4 K \times \mathbb A^2_{[x,y]} \to \standardTarget g 4 K.
\end{align*}
For $1 \leq i \leq 4$, define $\standardFamily g i K$, the \emph{standard families} of genus-$g$ hyperelliptic curves by completing the above smooth affine
curve over $\standardTarget g i K$ to a smooth projective curve over $\standardTarget g i K$. The definition of $\Delta^{(i)}$ ensures that these are indeed genus-$g$ hyperelliptic curves. 
For a $K$-valued point $u \in \standardTarget g i K (K)$, we denote by $A_u$ the Jacobian (which is necessarily a $g$-dimensional PPAV) of the fiber over $u$ of the corresponding standard family.
\end{definition}

As we show in Section~\ref{symbed},
the mod-$2$ Galois image of the Jacobian of a member of $\standardFamily{g}{i}{K}$ always lands in a certain copy of the symmetric group $S_{2g+2-(i \bmod 2)} \subset \Sp_{2g}(\ZZ/2\ZZ)$. Denote by $\grouphypboth{g}{(i \bmod 2)}{K}$ the intersection of the following two subgroups of $\GSp_{2g}(\wh{\ZZ})$: (1) the subgroup of those matrices with multiplier landing in $\chi(K) \subset \wh{\ZZ}^\times$, where $\chi$ denotes the cyclotomic character, and (2) the preimage of $S_{2g+2-(i \bmod 2)}$ under the projection map $\GSp_{2g}(\wh{\ZZ}) \to \Sp_{2g}(\ZZ/2\ZZ)$. 
Let $\on{Ht}\colon\mathbb P^r(\overline K) \rightarrow \mathbb R_{>0}$
denote the absolute multiplicative height on projective space, and define a height function $\| - \|$ on the lattice $\mathcal O^r_K$
sending $\left( t_1, \ldots, t_r \right) \mapsto \max_{\sigma,i}|\sigma(t_i)|$,
where $\sigma$ varies over all field embeddings $\sigma\colon K \hookrightarrow \mathbb C$.
Having fixed this notation, our first main theorem may be stated as follows:

\begin{theorem}\label{mainbldg}	
Let $B> 0$, $i \in \{1,2,3,4\}$, $g \geq 2$, and let $n$ be an arbitrarily positive integer. Let $\delta_\QQ = 2$, and let $\delta_K = 1$ for $K \neq \QQ$. Then $[ \grouphypboth g {(i \bmod 2)}{K} : \rho_{A_u}(G_K) ] \geq \delta_K$
for all $u \in \standardFamily g i K(K)$, 
and we have the following asymptotic statements, with the implied constants depending only on $n$, $g$, and $K$:
			\[
				\frac{|\{u \in \standardTarget g i K(\OO_K) : \lVert u \rVert \le B,\, [ \grouphypboth g {(i \bmod 2)}{K} : \rho_{A_u}(G_K) ] = \delta_K \}|}{|\{u \in \standardTarget g i K(\OO_K): \lVert u \rVert \le B\}|} = 1 + O((\log B)^{-n}),
			\]
\[
	\frac{|\{u \in \standardTarget g i K(K) : \on{Ht}(u) \leq B,\, [ \grouphypboth g {(i \bmod 2)}{K} : \rho_{A_u}(G_K) ] = \delta_K \}|}{|\{u \in \standardTarget g i K(K) : \on{Ht}(u) \le B\}|} = 1 + O((\log B)^{-n}). 
			\]
\noindent Furthermore, the statement above applies
if we take $i = 2$ and replace
$\standardTarget g 2 K$ by any rational family 
of hyperelliptic curves
dominating the moduli of hyperelliptic curves,
so long as the map to the moduli of hyperelliptic
curves has geometrically connected generic fiber.
\end{theorem}

The methods employed to prove density-$1$ results like Theorem~\ref{mainbldg} do not lend themselves well to the construction of explicit examples, which may be useful insofar as they can provide evidence in support of related conjectures. 
We now give two explicit examples, improving upon the examples of \cite{dooleyfat} and \cite{anni2016residual} mentioned in Section~\ref{subsection:intro-background}.
To the authors' knowledge,
these are the first examples of hyperelliptic curves in genus $g = 2$ and $3$
whose mod-$\ell$ monodromy is equal to 
$\GSp_{2g}(\bz/\ell\bz)$ when $\ell > 2$,
and equal to $S_{2g+2}$ when $\ell = 2$.
Moreover, we show the Galois representations of
these curves have index $2$ in the group
$\grouphyp g {\mathbb Q}$. Note that all hyperelliptic
curves over $\bq$ have Galois representation
strictly contained in $\grouphyp g {\mathbb Q}$,
as follows from
Corollary~\ref{monodromy-stack} (since the monodromy of any curve
is contained in that of the universal family).
Hence, our examples yield curves with
maximal monodromy among all hyperelliptic curves
of genus $2$ and $3$.

\begin{theorem}\label{exemplinongratia}
Let $C_2$ and $C_3$ over $\QQ$ be smooth projective models of the affine plane curves cut out by the equations
\begin{align*}
C_2 & : \quad y^2 = x^6 + 7471225x^5 + 16548721x^4 + 6639451x^3 + 16857421x^2 + \\
& \qquad \qquad \qquad 20754195x +9508695, \text{ and} \\
C_3 & : \quad y^2 = x^8 + 10781051650x^7 + 5302830080x^6 + 33362176x^5 + 10656581376x^4  + \\
& \qquad \qquad \qquad 5522318080x^3 + 4238752256x^2  + 3613465600x  + 3725404480.
\end{align*}
Then for each $g \in \{2, 3\}$, the Jacobian $J_{C_g}$ of $C_g$ is a $g$-dimensional PPAV over $\QQ$ satisfying the condition $[\grouphyp g {\mathbb Q} : \rho_{J_{C_g}}(G_\QQ)] = 2$. \end{theorem}
\begin{remark}
In checking the examples declared in Theorem~\ref{exemplinongratia}, we combined the methods developed in~\cite{anni2017constructing} and~\cite{seaweed} to expedite the verification process. It is also possible to modify the techniques introduced in~\cite{seaweed} to show that the curves cut out by the equations
\begin{align}
f(x) & = x^6 - 2x^4 - 2x^3 - 3x^2 - 2x + 1 \quad \text{and} \\
f(x) & = x^8 -4x^3 + 4x + 4,
\end{align}
which are of respective genera $2$ and $3$, both have maximal monodromy. The reader may contact any one of the authors if further details of the proof of this claim are desired.
\end{remark}

The rest of this paper is organized as follows:
Section~\ref{section:group-theory} is concerned with proving the
group-theoretic Theorem~\ref{theorem:small-ab}.
In Section~\ref{prelimswine},
we use Theorem~\ref{theorem:small-ab} to prove
Lemma~\ref{theorem:r=2}, which
is employed in the proof of 
Theorem~\ref{mainbldg} to verify the claimed value
of $\delta_K$.
In Section~\ref{subsection:monodromy-of-families}
we compute the monodromy of various families of hyperelliptic curves.
We combine these two results to prove Theorem~\ref{mainbldg}
in Section~\ref{32isnotagoodscoreline}.
Finally, in Section~\ref{verified},
we prove Theorem~\ref{exemplinongratia}.
