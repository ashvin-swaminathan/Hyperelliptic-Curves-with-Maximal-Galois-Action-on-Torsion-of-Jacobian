\section{Definitions and Properties of Symplectic Groups}
\label{section:group-theory}

This section is devoted to proving Theorem~\ref{theorem:small-ab},
which is needed for proving the main results of this paper, Theorems~\ref{mainbldg} and~\ref{exemplinongratia}. We start in Sections~\ref{subsection:stimpy} and~\ref{subsection:notation} by defining symplectic groups, discussing their basic properties, and introducing some recurring notation. Then, in Section~\ref{mygawdjamesyousabeast} we prove a result that is a crucial input to Section~\ref{section:proof-of-mainbldg}, where we prove Theorem~\ref{mainbldg}.
The reader may choose to continue directly to Section~\ref{section:proof-of-mainbldg} after studying the statement of Theorem~\ref{mygawdjamesyousabeast}.

\subsection{Symplectic Groups}\label{subsection:stimpy}

Let $R$ be a commutative ring, and let $g$ be a positive integer. Let $M$ be a free $R$-module of rank $2g$, and let $\langle -, - \rangle \colon M \times M \to R$ be a non-degenerate alternating bilinear form on $M$. Define the {\it general symplectic group} (otherwise known as the \emph{group of symplectic similitudes}) $\GSp(M) \subset \on{GL}(M)$ to be the subgroup of all $R$-automorphisms $S$ such that there exists some $m_S \in R^\times$, called the {\it multiplier} of $S$, satisfying $\langle S v, Sw \rangle = m_S \cdot \langle v, w \rangle$ for all $v, w \in M$. If $m_S$ exists, then it is necessarily unique, and one easily checks that the resulting {\it mult} map 
\begin{align*}
	\mult \colon \GSp(M) & \rightarrow R^\times \\
	S & \mapsto m_S
\end{align*}
is a group homomorphism; we call its kernel the {\it symplectic group} $\Sp(M)$.

Choose an $R$-basis for $M$, and denote by $\Omega_{2g}$ the matrix which expresses the inner product $\langle - , - \rangle$ with respect to this basis.
The choice of basis gives rise to an identification $\GL(M) \simeq \GL_{2g}(R)$, and we take $\GSp_{2g}(R)$ to be the image of $\GSp(M)$ and $\Sp_{2g}(R)$ to be the image of $\Sp(M)$ under this identification. Let $\det \colon \GL_{2g}(R) \to R^\times$ be the determinant map, and observe that the diagram
\begin{center}
\begin{tikzcd}
\GSp(M) \arrow{r}{\sim} \arrow[swap]{rd}{\on{mult}^g} &  \GSp_{2g}(R) \arrow{d}{\on{det}} \\
& R^\times
\end{tikzcd}
\end{center}

\noindent commutes. Note
that $\GSp_{2g}(R) \subset \GL_{2g}(R)$ is the subgroup of all invertible matrices $S$ satisfying $S^T \Omega_{2g} S = (\on{mult} S) \, \Omega_{2g}$ 
and that $\Sp_{2g}(R) = \ker(\on{mult} \colon \GSp_{2g}(R) \to R^\times)$.

Let $\on{Mat}_{2g \times 2g}(R)$ be the space of $2g \times 2g$ matrices having entries in $R$, and consider the Lie algebras $\mf{gsp}_{2g}(R)$ and $\mf{sp}_{2g}(R)$ defined by 
		\begin{align*}
        \mf{gsp}_{2g}(R) &\defeq \{\Lambda \in \on{Mat}_{2g \times 2g}(R) : \Lambda^T \Omega_{2g} + \Omega_{2g} \Lambda = d \cdot \Omega_{2g} \text{ for some } d \in R \}, \\
			\mf{sp}_{2g}(R) &\defeq \{\Lambda \in \on{Mat}_{2g \times 2g}(R) : \Lambda^T \Omega_{2g} + \Omega_{2g} \Lambda = 0 \}. 
		\end{align*} 
When studying Galois representations associated to PPAVs, we usually take $R$ to be one of the following: the profinite completion $\wh{\ZZ}$ of $\ZZ$, the ring of $\ell$-adic integers $\ZZ_{\ell}$ for a prime number $\ell$, or the finite cyclic ring $\ZZ / m \ZZ$ for a positive integer $m$. 
Observe that we have the following isomorphisms of topological groups:
\begin{equation}\label{orientation1}
\GSp_{2g}(\ZZ_\ell)   \simeq  \varprojlim_k \GSp_{2g}(\ZZ/ \ell^k \ZZ) \quad \text{and}
\end{equation}
\begin{equation}\label{orientation2}
\prod_{\text{prime } \ell} \GSp_{2g}(\ZZ_\ell) \simeq  \GSp_{2g}(\wh{\ZZ}) \simeq  \varprojlim_m \GSp_{2g}(\ZZ / m \ZZ).
\end{equation}
The isomorphisms \eqref{orientation1} and~\eqref{orientation2} remain valid if $\GSp_{2g}$ is replaced by $\Sp_{2g}$. As for the Lie algebras, note that by sending $\Lambda \mapsto \id_{2g} + \ell^k \Lambda$ we obtain group isomorphisms
\begin{align*}
\mf{gsp}_{2g}(\ZZ/ \ell \ZZ) &\simeq \ker(\GSp_{2g}(\ZZ/\ell^{k+1} \ZZ) \to \GSp_{2g}(\ZZ/ \ell^k \ZZ)), \\
\mf{sp}_{2g}(\ZZ/ \ell \ZZ) &\simeq \ker(\Sp_{2g}(\ZZ/\ell^{k+1} \ZZ) \to \Sp_{2g}(\ZZ/ \ell^k \ZZ))
\end{align*}
for every $k \geq 1$, so when it is useful or convenient, we will sometimes use the Lie algebra notation to denote the above kernels.


\subsection{Computing Commutators of Large Subgroups of $\GSp_{2g}(\ZZ_2)$} \label{mygawdjamesyousabeast}
The objective of this section is to prove a soon-to-be-useful theorem concerning the commutator of a subgroup of $\GSp_{2g}(\ZZ_2)$ which is the preimage (under mod-2 reduction) of a subgroup of $\Sp_{2g}(\bz / 2 \bz)$ that contains a copy of the symmetric group $S_{2g+1}$.

\subsubsection{Embedding the Symmetric Group, Take 1} \label{take1}

We asserted in the discussion immediately preceding the statement of Theorem~\ref{mainbldg} that the symmetric group $S_{2g+1}$ may be viewed as a subgroup of $\Sp_{2g}(\ZZ/2\ZZ)$. We now provide a working description of the way in which this embedding is constructed; the manner in which this description applies to the context of studying hyperelliptic curves is discussed in Section~\ref{symbed}.
\begin{lemma} \label{lemma:include-s}
For every $g \ge 2$, we have an inclusion $S_{2g+2} \hookrightarrow \Sp_{2g}(\bz /2 \ZZ)$.
When $g = 2$, this inclusion is an isomorphism. 
\end{lemma}
\begin{proof} 
	Let $V$ be a $(2g+2)$-dimensional vector space over $\FF_2$, and equip $V \simeq \mathbb{F}_2^{2g+2}$ with the standard inner product. Let $t \defeq (1, \ldots, 1)$ be the vector whose components are all equal to $1$. Then the hyperplane $t^\perp \subset V$ of all vectors orthogonal to $t$ actually contains $t$ since $\dim V = 2g+2$ is even. Moreover, if we define $W = t^\perp / \on{span}(t)$, the inner product on $V$ descends to a nondegenerate alternating bilinear form on $W$. The action of $S_{2g+2}$ given by permuting the coordinates of $V$ fixes both $t$ and $t^\perp$, so it descends to an action on $W$ that preserves the bilinear form. 
Thus, we obtain an inclusion of $S_{2g+2}$ into the group of symplectic transformations of $W$ with multiplier $1$. For a more conceptual
explanation of this inclusion in terms of the two-torsion of hyperelliptic
curves, see Section \ref{symbed}.
Upon choosing a suitable basis for $W$ we may identify this group with $\Sp_{2g}(\ZZ / 2 \ZZ)$. For $g = 2$, the resulting inclusion is an isomorphism because $\#(S_6) = 720 = \#(\Sp_4(\ZZ / 2 \ZZ))$.
\end{proof}
We embed $S_{2g+1} \hookrightarrow S_{2g+2}$ as the subgroup fixing the vector $(0,\dots,0,1) \in \FF_2^{2g+2}$.

\subsubsection{Notation}
\label{subsection:notation}

In what follows, we shall (for the most part) study subquotients of $\GSp_{2g}(\ZZ_2)$ and $\GSp_{2g}(\ZZ/ 2^k \ZZ)$ for $k$ a positive integer. We employ the following notational conventions:
\begin{itemize}
\item Let $H\subset \GSp_{2g}(\ZZ_2)$ be a closed subgroup.
\item For $m, n \in \ZZ_{> 0} \cup \{\infty\}$ 
		with $m > n$, let $\gify{2^m}{2^n} \colon \GSp_{2g}(\ZZ/2^m \ZZ) \to \GSp_{2g}(\ZZ/2^n \ZZ)$ and $\fy{2^m}{2^n} \colon \Sp_{2g}(\ZZ/2^m \ZZ) \to \Sp_{2g}(\ZZ/2^n \ZZ)$ be the natural projection maps. (When $m = \infty$, $\ZZ/2^m\ZZ$ denotes $\ZZ_2$.) 
\item Let $H(2^k) = \gify{2^\infty}{2^k}(H) \subset \GSp_{2g}(\ZZ/2^k \ZZ)$ be the mod-$2^k$ reduction of $H$.
\item For any topological group $G$, let $[G,G]$ be the closure of its commutator subgroup, and let $G^{\ab} \defeq G/{ {[G,G]}}$ be its abelianization.
\item For each positive integer $n$, let $\id_n$ denote the $n \times n$ identity matrix.
\end{itemize}

\subsubsection{Main Group Theoretic Result}

We can now state the main theorem of this section.

\begin{theorem} \label{theorem:small-ab}
	Let $g \geq 2$. Let $H \subset \GSp_{2g}(\bz_2)$ be a subgroup such that $H = \gify{2^\infty}{2}^{-1}(H(2))$    and such that $H(2)$ contains $S_{2g+1}$. 
    Then we have that
	\begin{equation}\label{thefirstclaim}
    [H, H] = \fy{2^\infty}{2}^{-1}([H(2), H(2)]).
    \end{equation}
	Moreover, the homomorphism \(H \to (H(2))^{\ab} \times (\bz_2)^\times\), defined on the left component by postcomposing reduction mod-2 with the abelianization map $H(2) \to H(2)^{\on{ab}}$ and on the right component by the multiplier map $\on{mult}$, induces an isomorphism
    \begin{equation}\label{thesecondclaim}
    H^{\on{ab}} \simeq (H(2))^{\ab} \times (\bz_2)^\times.
    \end{equation}
\end{theorem}

The relevance of Theorem~\ref{theorem:small-ab} to studying Galois representations of Jacobians of hyperelliptic curves is described in Lemma~\ref{theorem:r=2}, given at the beginning of Section~\ref{section:proof-of-mainbldg}.
We prove Theorem~\ref{theorem:small-ab} next in Section~\ref{subsection:proof-of-first-claim}.

\subsection{Proof of Theorem~\ref{theorem:small-ab}}
\label{subsection:proof-of-first-claim}
\begin{proof}[Proof of Theorem~\ref{theorem:small-ab} assuming
	Corollary~\ref{lemma:contains-mod-4} and 
Proposition~\ref{lemma:contains-mod-2}]
Because we have that $$[H, H](2) = [H(2), H(2)],$$ in order to prove~\eqref{thefirstclaim}, it suffices to prove that 
\(
	[H, H] \supset \ker \fy{2^\infty}{2}. 
\)
To prove this statement, it further suffices to prove the following two statements:
\begin{enumerate}
\item[\customlabel{property-a}{(A)}] \(
	[H, H] \supset \ker \fy{2^\infty}{4},
\)
\item[\customlabel{property-b}{(B)}] \(
	[H(4), H(4)] \supset \ker \fy{4}{2}.
\)
\end{enumerate}
Statement~\ref{property-a} is proven in 
Corollary~\ref{lemma:contains-mod-4}
and statement~\ref{property-b} is proven in
Proposition~\ref{lemma:contains-mod-2}.
To complete the proof, we only need verify~\eqref{thesecondclaim}. Note that~\eqref{thefirstclaim} tells us that the map $H \to (H(2))^{\ab} \times (\bz_2)^\times$ has kernel precisely $[H,H]$, so to prove~\eqref{thesecondclaim}, it suffices to check that the map $H \to (H(2))^{\ab} \times (\bz_2)^\times$ is surjective. 
But this is easy to check by hand: 
For $\alpha \in (\mathbb Z_2)^\times$, let $N_\alpha$ be the matrix which
has alternating $1$'s and $\alpha$'s on the diagonal, taken with respect to
a symplectic basis $e_1, \ldots, e_{2g}$ where $\langle e_{i}, e_{j} \rangle$ 
is $1$ if $i = 2k, j = 2k+1$ for some integer $k$, is $-1$ if $i = 2k+1, j = 2k$, and is $0$ otherwise.
For $(M_2, \alpha) \in (H(2))^{\ab} \times (\bz_2)^\times$, let $M_2^\infty \in \fy{2^\infty}{2}^{-1}(M_2)$, and observe that $M_2^\infty \cdot N_\alpha \mapsto (M_2, \alpha)$ via the map $H \to (H(2))^{\ab} \times (\bz_2)^\times$. This concludes the proof of our main group-theoretic result, Theorem~\ref{theorem:small-ab}.
\end{proof}
\subsubsection{Proving Statement~\ref{property-a}}\label{honeyseek}

We begin with the following lemma, in which we compute the commutator subalgebra of $\mf{gsp}_{2g}(\ZZ / 2 \ZZ)$.

\begin{lemma} \label{lemma:gsp-perfect} 
	Let $\ell$ be a prime number. We have
	\(
		[\mf{gsp}_{2g}(\bz / \ell \bz), \mf{gsp}_{2g}(\bz / \ell \bz)] = \mf{sp}_{2g}(\bz / \ell \bz).
	\)\footnote{This result is a variant of~\cite[Proposition 2.10]{landesman-swaminathan-tao-xu:rational-families}.}
\end{lemma} 

\begin{proof} 
For convenience, let $\mf{g}_\ell \defeq [\mf{gsp}_{2g}(\bz / \ell \bz), \mf{gsp}_{2g}(\bz / \ell \bz)]$. That $\mf{g}_\ell \subset \mf{sp}_{2g}(\ZZ/\ell\ZZ)$ is obvious from the definitions of $\mf{gsp}_{2g}(\ZZ/\ell\ZZ)$ and $\mf{sp}_{2g}(\ZZ/\ell\ZZ)$, so it suffices to prove that reverse containment. For $\ell \geq 3$, this is immediate from~\cite[Proposition 2.10]{landesman-swaminathan-tao-xu:rational-families}, so we may restrict to the case where $\ell = 2$ (note that this is the case of primary interest to us).\footnote{In essence, the reason why the case of $\ell$ odd needs to be handled separately is that $\mf{sp}_{2g}(\ZZ/\ell\ZZ)$ is a perfect Lie algebra if and only if $\ell$ is odd, a result due to Hogeweij~\cite{hog1982}.} Choose a basis for $(\ZZ/2\ZZ)^{2g}$ with respect to which $\Omega_{2g}$ is given by
    \(
    	\Omega_{2g} = \left[ \begin{array}{c|c} 0 & \id_g \\  \hline  -\id_g & 0 \end{array} \right].
    \)
    Then $\mf{sp}_{2g}(\bz / 2 \bz)$ consists of matrices of the form
    \(
   	\left[ \begin{array}{c|c} A & B \\ \hline  C & -A^T \end{array} \right]
    \)
   where $A,B,C \in \on{Mat}_{g \times g}(\ZZ/2\ZZ)$ and $B,C$ are required to be symmetric. Since we have
\begin{align}
	\label{equation:block-diagonal-commutator}
 \left[ \left[\begin{array}{c|c} A & 0 \\ \hline 0 & -A^T \end{array}\right], \left[\begin{array}{c|c} D & 0 \\ \hline 0 & -D^T \end{array}\right] \right] & = \left[\begin{array}{c|c} AD - DA & 0 \\ \hline 0 & A^TD^T - D^TA^T \end{array}\right],
 \intertext{all block-diagonal matrices in $\mf{sp}_{2g}(\ZZ/2\ZZ)$ with every diagonal entry equal to $0$ are contained in $\mf{g}_2$. Moreover, for symmetric $B,C,E,F \in \on{Mat}_{g \times g}(\ZZ/2\ZZ)$, we have}
\label{equation:block-off-diagonal-commutator}
 \left[ \left[\begin{array}{c|c} 0 & B \\ \hline C & 0 \end{array}\right], \left[\begin{array}{c|c} 0 & E \\ \hline F & 0 \end{array}\right] \right] & = \left[\begin{array}{c|c} BF - EC & 0 \\ \hline 0 & CE - FB \end{array}\right],
 \intertext{and we can arrange that $BF-EC$ is an elementary matrix with a single nonzero entry on the diagonal.
	 Summing matrices from~\eqref{equation:block-diagonal-commutator} and~\eqref{equation:block-off-diagonal-commutator},
tells us that all block-diagonal matrices in $\mf{sp}_{2g}(\ZZ/2\ZZ)$ are contained in $\mf{g}_2$. Additionally, note that $\mf{gsp}_{2g}(\bz / 2 \bz)$ also contains
   	\(\left[ \begin{array}{c|c} \id_g & 0 \\ \hline  0 & 0  \end{array} \right], \)
    from which we deduce that $\mf{g}_2$ contains }
	    	\left[ \left[ \begin{array}{c|c} \id_g & 0 \\ \hline  0 & 0  \end{array} \right], \left[ \begin{array}{c|c} 0 & B \\ \hline  0 & 0 \end{array} \right] \right] &= \left[ \begin{array}{c|c} 0 & B \\ \hline  0 & 0 \end{array} \right], 
\end{align}
where $B \in \on{Mat}_{g \times g}(\ZZ/2\ZZ)$ is symmetric. One similarly checks that $\mf{g}_2$ contains 
\(\left[ \begin{array}{c|c} 0 & 0 \\ \hline  C & 0 \end{array} \right],\)
for $C \in \on{Mat}_{g \times g}(\ZZ/2\ZZ)$ symmetric. It follows that $\mf{g}_2 \supset \mf{sp}_{2g}(\ZZ/2\ZZ)$. \qedhere
\end{proof} 

\begin{corollary} \label{corollary:coolwhip} 
	We have
	\(
		[\ker \gify{2^\infty}{2}, \ker \gify{2^\infty}{2}] = \ker \fy{2^\infty}{4}.
	\)
\end{corollary} 
\begin{proof} 
Clearly \([\ker \gify{2^\infty}{2}, \ker \gify{2^\infty}{2}] \subset \ker \fy{2^\infty}{4}\), so it suffices to prove the reverse inclusion. By~\cite[Lemma 2.11]{landesman-swaminathan-tao-xu:rational-families}, we have that $[\ker \gify{2^\infty}{2}, \ker \gify{2^\infty}{2}] \supset \ker \fy{2^\infty}{8}$. Then, identifying $\mf{gsp}_{2g}(\ZZ/ 2 \ZZ)$ with $\ker \gify{8}{4}$ and $\mf{sp}_{2g}(\ZZ/2\ZZ)$ with $\ker \fy{8}{4}$, we have by Lemma~\ref{lemma:gsp-perfect} that
\begin{align*}
\ker \fy{8}{4} & = \left[ \ker \gify{8}{4},\ker \gify{8}{4} \right] = \left[ \ker \gify{2^\infty}{4},\ker \gify{2^\infty}{4} \right](8) \\
& \subset \left[ \ker \gify{2^\infty}{2},\ker \gify{2^\infty}{2} \right](8)
\end{align*}
It follows that $\left[ \ker \gify{2^\infty}{2},\ker \gify{2^\infty}{2} \right] \supset \ker \fy{2^\infty}{4}$.
\end{proof} 

\begin{corollary} \label{lemma:contains-mod-4}
	We have $\ker \fy{2^\infty}{4} \subset [H, H]$. 
\end{corollary}
\begin{proof} 
	The hypothesis that 
    \(
    	H = \gify{2^\infty}{2}^{-1}(H(2))
    \)
    implies that $\ker \gify{2^\infty}{2} \subset H$, and hence $[\ker \gify{2^\infty}{2}, \ker \gify{2^\infty}{2}] \subset [H, H]$. Applying Corollary~\ref{corollary:coolwhip} then yields the desired result.
\end{proof} 

\subsubsection{Proving Statement (B): Tensor Product Notation}\label{tomswifties}

Just as we did in the proof of Lemma~\ref{lemma:gsp-perfect}, we must choose a basis with respect to which our symplectic form has an easy-to-use matrix representation. The goal of this subsection is to choose such a basis and to develop a shorthand notation for this basis. In Sections~\ref{subsection:s-action} and~\ref{subsection:finishing-the-proof}, we use this notation to prove Statement (B), thereby completing the proof of Theorem~\ref{theorem:small-ab}.

Recall the notation introduced in the first paragraph of Section~\ref{subsection:stimpy}: $R$ is a commutative ring (which we will take to be either $\ZZ_2$ or $\ZZ/4\ZZ$), and $M$ is a free $R$-module of rank $2g$. We choose a basis $(e_1, \ldots, e_{2g})$ for $M$ so that the symplectic form $\langle -, -\rangle$ is given by
$$\langle e_i , e_j \rangle \defeq \begin{cases} j-i & \text{ if $|j-i| = 1$ and $\max\{i,j\} \equiv 0\,(\bmod\,2$) } \\ 0 & \text{ otherwise} \end{cases}$$

We may alternatively construct $M$ as follows. Let $N_1 \simeq R^{2}$ have basis $(x_1, x_2)$ and let $N_2 \simeq R^{g}$ have basis $(y_1, \ldots, y_g)$. Endow $N_1$ with the alternating form given by $\langle x_i, x_j \rangle = j - i$, and endow $N_2$ with the symmetric form given by $\langle y_i, y_j \rangle = \delta_{ij}$, where $\delta_{ij}$ denotes the Kronecker $\delta$-function as usual. Then if we take $M \defeq N_1 \otimes N_2$, we have that $(x_i \otimes y_j : i \in \{1,2\}, \, j \in \{1, \dots, g\} )$ is a basis for $M$ and that $M$ is equipped with an alternating form defined on simple tensors by
\[
\langle a_1 \otimes b_1, a_2 \otimes b_2 \rangle \defeq \langle a_1, a_2 \rangle \cdot \langle b_1, b_2 \rangle. 
\]
Note that the map sending $x_i \otimes y_j \mapsto e_{2j + i - 2}$ gives an identification between our two different constructions of $M$.

Linear operators on $M$ are $R$-linear combinations of tensor products of linear operators on $N_1$ with linear operators on $N_2$. If we denote by $x_{ij}$ the row-$i$, column-$j$ elementary matrix acting on $N_1$ and by $y_{mn}$ the row-$m$, column-$n$ elementary matrix acting on $N_2$, then a basis for $\on{End}(M)$ is given by $(x_{ij} \otimes y_{mn} : i,j \in \{1, 2\},\, m,n \in \{1, \dots, g\})$. Also notice that any element $\Lambda \in \on{End}(M)$ may be expressed as
\begin{equation}\label{fouseytattoo}
\Lambda =  \sum_{i=1}^g \sum_{j=1}^g \Lambda_{ij} \otimes y_{ij}
\end{equation}
where $\Lambda_{ij} \in \on{End}(N_1)$ for all $i,j \in \{1,\dots, g\}$.
\begin{proposition} \label{proposition:lie}
	Let $\phi \in \on{End}(\on{End}(N_1))$ be defined by
	\[
	x_{11} \mapsto -x_{22}, \quad x_{22} \mapsto -x_{11}, \quad x_{12} \mapsto x_{12}, \quad x_{21} \mapsto x_{21}.
	\]
The Lie algebra $\mf{gsp}_{2g}(R)$ consists of those elements $\Lambda \in \on{End}(M)$ with $\Lambda_{ij} \in \on{End}(N_1)$ such that there exists $d \in R$ satisfying
    $$\phi(\Lambda_{ji}) = \Lambda_{ij} - (d\delta_{ij}) \cdot \id_2.$$
Moreover, $\mf{sp}_{2g}(R)$ admits an analogous description in which $d$ is required to be zero. 
\end{proposition}
\begin{proof}  
Since $\Omega_{2g}^2 = -\id_{2g}$, the defining equation for $\mf{gsp}_{2g}(R)$ is equivalent to
\begin{align}	
\Omega_{2g} \Lambda^T \Omega_{2g} - \Lambda + d \cdot (\id_2 \otimes \id_g) & = 0. \label{equation:lie-algebra-condition}
\end{align} 
Note that the identity element $\on{End}(N_2)$ is given by $\id_g = y_{11} + \cdots + y_{gg}$. Substituting this in along with the expansion~\eqref{fouseytattoo} for $\Lambda$ as well as $(x_{12} - x_{21}) \otimes \id_g$ for $\Omega_{2g}$ on the left-hand side of~\eqref{equation:lie-algebra-condition} yields that
\[
\sum_{i =1}^g \sum_{j=1}^g \big[ (x_{12} - x_{21})(\Lambda_{ji})^T(x_{12} - x_{21}) - \Lambda_{ij} + (d\delta_{ij}) \cdot \id_2 \big] \otimes y_{ij} = 0,
\]
which is equivalent to the following condition:
\[
(x_{12} - x_{21})(\Lambda_{ji})^T(x_{12} - x_{21}) = \Lambda_{ij} - (d\delta_{ij}) \cdot \id_2
\]
The desired result then follows upon observing that 
	\(
	(x_{12} - x_{21})(\Lambda_{ji})^T(x_{12} - x_{21}) = \phi(\Lambda_{ji}). \qedhere
	\)
\end{proof} 

\begin{remark} \label{remark:sp-basis}
	When $R = \ZZ/2\ZZ$, minus signs may be ignored, so the operator $\phi$ may be concisely described as transposition across the anti-diagonal. It follows from Proposition~\ref{proposition:lie} that the following is a basis for $\mf{sp}_{2g}(\ZZ/2\ZZ)$:
	\begin{align*}
    & \big(\id_2 \otimes y_{ii}, x_{12} \otimes y_{ii}, x_{21} \otimes y_{ii} : i \in \{1, \dots, g\}\big)\,\, \cup \\
    & \big(x_{12} \otimes (y_{ij}+y_{ij}) , x_{11} \otimes y_{ij} + x_{22} \otimes y_{ji} ,  x_{21} \otimes (y_{ij} + y_{ji}) , x_{22} \otimes y_{ij} + x_{11} \otimes y_{ji} : 1 \leq i < j \leq g \big).
\end{align*}
In Section~\ref{subsection:finishing-the-proof}, it will be convenient to define a function $\on{ind}$ that assigns to each of the above basis elements the value of $i$ (e.g., $\on{ind}(\id_2 \otimes y_{ii}) = i$ and $\on{ind}(x_{12} \otimes (y_{ij} + y_{ij})) = i$).
\end{remark}  

\subsubsection{Proving Statement (B): Describing the Action of $S_{2g+2}$} \label{subsection:s-action} 

We now seek to describe the embedding $S_{2g+2} \hookrightarrow \Sp_{2g}(\bz / 2 \bz)$ from Lemma~\ref{lemma:include-s} in terms of the tensor product notation that we just introduced in Section~\ref{tomswifties}. To this end, we set $R = \ZZ/2\ZZ$, so that $M \simeq \FF_2^{2g}$.
\begin{lemma} \label{lemma:basis} 
	Recall notation from the proof of Lemma~\ref{lemma:include-s}. The map $\psi: M \ra t^\perp / \langle t \rangle$ of symplectic vector spaces defined by
	\[ x_1 \otimes y_n \mapsto \sum_{i=1}^{2n} e_i \quad \text{and} \quad
	x_2 \otimes y_n \mapsto e_{2n+1} + \sum_{i=1}^{2n-1} e_i \quad \text{for each} \quad n \in \{1, \dots, g\}
	\] 
    is an isomorphism.
\end{lemma} 
\begin{proof}
The lemma follows immediately from the observation that $\psi$ identifies the symplectic forms of $M$ and $t^\perp / \langle t \rangle$.
\end{proof}
Recall that the group $S_{2g+2}$ is generated by the adjacent transpositions $T_k$ for $k \in \{1, \dots, 2g+1\}$ whose cycle types are given by $T_k = (k, k+1)$. We now compute the action of $T_k$ on $M$ for each $k$: 
\begin{lemma} \label{lemma:s-action}
	When viewed as operators on $M$, the transpositions $T_k$ are given by
	\begin{align*} 
	T_{2n} &= \id_{2g} + (x_{11} + x_{12} + x_{21} + x_{22}) \otimes y_{nn}, \\
	T_{2n+1} &= \id_{2g} + x_{12} \otimes (y_{nn} + y_{(n+1) n} + y_{n (n+1)} + y_{(n+1) (n+1)}), 
	\end{align*}
	according as $k = 2n$ or $k = 2n+1$, where any term with an out-of-range index is zero. 
\end{lemma} 
\begin{proof} 	
	The result for $k = 2n$ follows from the observation that $T_{2n}$ swaps $x_1 \otimes y_n$ with $x_2 \otimes y_n$ and keeps all the other basis vectors fixed. As for $k = 2n+1$, we break into three cases:  
	\begin{enumerate} 
		\item Suppose $n = 0$. The transposition $T_1$ sends 
		$$\psi(x_2 \otimes y_1) = ( e_1 + e_3)  \mapsto (e_1 + e_2) + (e_1 + e_3) 
		= \psi(x_1 \otimes y_1) + \psi(x_2 \otimes y_2)$$
		and fixes all other $\psi(x_i \otimes y_j)$. Thus, $T_1$ is given by 
		\(
		T_1 = \id_{2g} + x_{12} \otimes y_{11}. 
		\)
		\item Suppose $n = g$. The transposition $T_{2g+1}$ sends 
		$$ \psi(x_2 \otimes y_g) = e_{2g + 1} + \sum_{i=1}^{2g-1} e_i  \mapsto  \sum_{i=1}^{2g+1} e_i + \sum_{i=1}^{2g-1} e_i  = e_{2g} + e_{2g+1} = \psi(x_1 \otimes y_g) + \psi(x_2 \otimes y_g) $$
		and fixes all other $\psi(x_i \otimes y_j)$. Thus, $T_g$ is given by 
		\(
		T_{2g+1} = \id_{2g} + x_{12} \otimes y_{gg}. 
		\)
		\item Finally, suppose $n \in \{1, \dots, g-1\}$. The transposition $T_{2n+1}$ sends 
		\begin{align*}
		\phi(x_2 \otimes y_n) = e_{2n + 1} + \sum_{i=1}^{2n-1} e_i \,\,\, & \mapsto \,\,\, \sum_{i=1}^{2n+2} e_i + \left( e_{2n+1} + \sum_{i=1}^{2n-1} e_i \right) + \sum_{i=1}^{2n} e_i \\
		&\hphantom{\quad\mapsto}= \psi(x_1 \otimes y_{n+1}) + \psi(x_2 \otimes y_n) + \psi(x_1 \otimes y_n), \\
		\psi(x_2 \otimes y_{n+1}) 
		=  e_{2n + 3} + \sum_{i=1}^{2n+1} e_i \,\,\, &\mapsto \,\,\, e_{2n+3} + \sum_{i=1}^{2n+1} e_i + \sum_{i=1}^{2n+2} e_i + \sum_{i=1}^{2n} e_i \\
		&\hphantom{\quad\mapsto}= \psi(x_2 \otimes y_{n+1}) + \psi(x_1 \otimes y_{n+1}) + \psi(x_1 \otimes y_n), 
		\end{align*} 
		and fixes all other $\psi(x_i \otimes y_n)$. Thus, $T_{2n+1}$ is given by 
		\[
		T_{2n+1} = \id_{2g} + x_{12} \otimes (y_{nn} + y_{(n+1) n} + y_{n (n+1)} + y_{(n+1) (n+1)}). 
		\]
	\end{enumerate} 
	The result for $k = 2n+1$ follows immediately from points (1)--(3) above.
\end{proof} 

\subsubsection{Finishing the Proof of Statement (B)} 	
\label{subsection:finishing-the-proof}
\begin{proposition} \label{lemma:contains-mod-2}
	We have $[H(4), H(4)] \supset \ker \fy{4}{2}$. 
\end{proposition} 
\begin{proof} 
	The assumption that 
    \(
    	H = \gify{2^\infty}{2}^{-1}(H(2))
    \)
    implies that 
	\(
	\ker \gify{4}{2} \subset H(4). 
	\)
	Recall that we may identify $\ker \gify{4}{2}$ with $\mf{gsp}_{2g}(\ZZ/2\ZZ)$, so that each $S \in \ker \gify{4}{2}$ may be expressed as $S = \id_{2g} + 2\Lambda$ where $\Lambda \in \mf{gsp}_{2g}(\bz / 2 \ZZ)$. The assumption that $H(2)$ contains $S_{2g+1}$ tells us that for any $M_2 \in S_{2g+1} \subset \Sp_{2g}(\bz / 2 \bz)$, we may lift $M_2$ to an element $M_4 \in H(4)$. In particular, we have that  
$$\id_{2g} + 2(\Lambda + M_2 \Lambda M_2^{-1}) = (\id_{2g} + 2\Lambda)^{-1}M_4(\id_{2g} + 2\Lambda)M_4^{-1} \in [H(4), H(4)].$$
To complete the proof, it suffices to show that matrices of the form $\Lambda + M_2 \Lambda M_2^{-1}$ span all of $\mf{sp}_{2g}(\bz / 2\bz)$. Let $V = \on{span}(\Lambda + M_2 \Lambda M_2^{-1} : \Lambda \in \mf{gsp}_{2g}(\ZZ/2\ZZ) \text{ and } M_2 \in S_{2g+1})$.
	
	It suffices to restrict our consideration to matrices $M_2$ corresponding to transpositions $T_k \in S_{2g+1}$. Note that $T_k = (T_k)^{-1}$, so that if we write $T_k = \id_{2g} + N_k$, then we have
    \begin{align} \label{equation:nuns-on-the-run}		
	\Lambda + T_k\Lambda(T_k)^{-1} = N_k\Lambda + \Lambda N_k + N_k \Lambda N_k.
	\end{align}	
	As in Lemma~\ref{lemma:s-action}, we will have to treat the cases $k = 2n$ and $k = 2n+1$ separately. In what follows, we induct on the value of the $\on{ind}$ function that $V$ contains the seven types of basis elements listed in Remark~\ref{remark:sp-basis}.
   First, however, we perform some calculations that serve to greatly simplify this inductive argument. Combining Lemma~\ref{lemma:s-action} with~\eqref{equation:nuns-on-the-run} and taking $\Lambda = x_{11} \otimes \id_g$, we find that 
\begin{align}
\Lambda + T_{2n} \Lambda T_{2n} & = \id_2 \otimes y_{nn} \in V, \label{jc1}\\
\Lambda + T_{2n-1} \Lambda T_{2n-1} & = x_{12} \otimes (y_{(n-1)(n-1)} + y_{(n-1)n} + y_{n(n-1)} + y_{nn}) \in V. \label{jc2} \\
\intertext{Repeating this calculation for $k = 2n$ but taking $\Lambda = x_{12} \otimes y_{nn}$, we find that}
	\Lambda + T_{2n} \Lambda T_{2n} & = (x_{12} + x_{21}) \otimes y_{nn} \in V. \label{jc9} \\
\intertext{Now fix $n, \ell$ with $\ell > n$, and take $\Lambda = M \otimes y_{n\ell} + \phi(M) \otimes y_{\ell n}$ for any $M \in \on{Mat}_{2 \times 2}(\bz / 2 \bz)$. By Proposition~\ref{proposition:lie}, all such $\Lambda$ are elements of $\mf{gsp}_{2g}(\bz/2\bz)$. We find that}
\Lambda + T_{2n} \Lambda T_{2n} & = (x_{11} + x_{12} + x_{21} + x_{22})M \otimes y_{n\ell} \, +  \label{jc5}\\
& \hphantom{===} \phi(M)(x_{11} + x_{12} + x_{21} + x_{22}) \otimes y_{\ell n} \in V, \nonumber \\ 
\Lambda + T_{2n-1}\Lambda T_{2n-1} & = x_{12}M \otimes (y_{n \ell} + y_{(n-1)\ell}) + \phi(M) x_{12} \otimes (y_{\ell n} + y_{\ell (n-1)}) \in V \label{jc3}. \\
\intertext{Taking $M = x_{11}$ in~\eqref{jc5}, so that $\phi(M) = x_{22}$, yields that}
	& (x_{11} + x_{21}) \otimes y_{n \ell} + (x_{21} + x_{22}) \otimes y_{\ell n} \in V, \label{jc6} \\
    \intertext{and taking $M = x_{22}$ in~\eqref{jc5}, so that $\phi(M) = x_{11}$, yields that}
	& (x_{12} + x_{22}) \otimes y_{n\ell} + (x_{11} + x_{12}) \otimes y_{\ell n} \in V. \label{jc7} \\
\intertext{Taking $M = x_{22}$ in~\eqref{jc3}, so that $\phi(M) = x_{11}$, yields that}
& x_{12} \otimes (y_{n \ell} + y_{\ell n}) \in V \label{jc4} \\
\intertext{and taking $M = x_{21}$ in~\eqref{jc4}, so that $\phi(M) = x_{21}$, yields that}
&  x_{11} \otimes y_{n \ell} + x_{22} \otimes y_{\ell n} \in V. \label{jc8}
\end{align}
We are now ready to carry out the induction. For the base case, we need to check that all basis vectors with $\on{ind}$-value equal to $1$ are in $V$; this follows immediately upon taking $n = 1$ in~\eqref{jc1}--\eqref{jc8}. Next, suppose for some $N \in \{1, \dots, g\}$ we have that all basis vectors with $\on{ind}$-value less than $N$ are in $V$. Taking $n = N$ in~\eqref{jc1}--\eqref{jc8} and applying the inductive hypothesis yields that all basis vectors with $\on{ind}$-value equal to $N$ are in $V$.
\end{proof}
