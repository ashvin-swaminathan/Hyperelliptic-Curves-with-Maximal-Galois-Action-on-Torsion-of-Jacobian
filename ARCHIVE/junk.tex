We know by either~\cite[Lemma 8.12]{mumford:tata-lectures-on-theta-ii} or~\cite[Th\'eor\`eme 1]{acampo:tresses-monodromie-et-le-groupe-symplectique},
	that 
	\begin{align*}
		\left\{
	M \in \Sp_{2g}(\widehat{\mathbb Z}) :M \equiv \id_{2g} \bmod 2
\right\} \subset \mono_{\standardFamily g 1 {\overline K}}.
	\end{align*}
	Recall that after restricting $\sigma_1: \symmetricFamily g 1 K \rightarrow \standardFamily g 1 K$ 
	to an open subscheme of the target,
	$\sigma_1$ is a covering map of degree $(2g+2)!$ by
	Lemma~\ref{lemma:connected-generic-fiber}.
	Therefore, the index $\left[ \mono_{\standardFamily g 1 K} : \mono_{\symmetricFamily g 1 {\overline K}} \right] \leq (2g+2)!$.
	\todo{This is clearly true, but is the justification sufficient?}
	\todo{An alternate proof, as suggested by james, would be to simply observe that there is a topological path in the the standard family which interchanges the roots, and this gives the full symmetric group, once we identify the
	action of the symmetric group with the roots of the polynomial}

	In order to complete the proof it suffices to show
	$\grouphyp g {\overline K} \subset \mono_{\standardFamily g 1 {\overline K}}$, since $\left[  \grouphyp g {\overline K} : \mono_{\standardFamily g 1 K} \right] = (2g+2)!$.
In turn to show this, it suffices to show that
$S_{2g+2}$ is contained in $\mono_{\standardFamily g 1 {\overline K}} \bmod 2$.
Since $\standardFamily g 1 {\overline K}$ is the base change of
$\standardFamily g 1 K$ to ${\overline K}$
and there is some finite extension $L/K$ so that
$\mono_{\standardFamily g 1 L} \bmod 2$
agrees with $\mono_{\standardFamily g 1 {\overline K}} \bmod 2$ 
it suffices to show that for any number field $L$ we have
$\mono_{\standardFamily g 1 L} \bmod 2 \supset S_{2g+2}$.
	
\todo{possibly make the following into a lemma if used elsewhere}
To see this, it in turn suffices to exhibit a single hyperelliptic
curve $C/L$ so that the mod-$2$ image of the Galois representation of this curve
contains $S_{2g+2}$.
Now, take any degree $2g+2$ polynomial $f(x)$ whose splitting field has
Galois group
$S_{2g+2}$ over $L$,
and consider the curve $y^2 = f(x)$.
Note such an $f(x)$ exists by Hilbert irreducibility.
Then, since the roots of $f(x)$ can be identified with the two-torsion
of the Jacobian of $C$, the action of $G_L$ on the roots of $f(x)$
can be identified with the mod-$2$ Galois representation.
Since the the Galois group of the splitting field of $f(x)$ is $S_{2g+2}$,
it follows that the mod-$2$ image of the Galois representation of $C$
is also $S_{2g+2}$, as desired.




\begin{proposition}
	\label{proposition:monodromy-of-small-standard-family}
	For $i \in \left\{ 3,4 \right\}$, we have $\mono_{\standardFamily g K i} = \grouphypboth g {(i-2)} K$.
\end{proposition}
\begin{proof}
	Since $i = 3,4$ are quite analogous, we only prove the slightly
	trickier case that $i = 3$.
	As in the proof of
	Lemma~\ref{lemma:geometric-monodromy-of-standard-family},
	we may reduce verifying the claim for geometric monodromy.
	
	We first show that the monodromy of
	$\symmetricFamily g 3 {\overline K}$ is precisely
	\begin{align*}
		\Gamma_2 := \left\{ M \in \Sp_{2g+2} : M \equiv I_{2g} \bmod 2 \right\}.
	\end{align*}			
	Since the action of $\pi_1(\symmetricFamily g {\overline K})$ on the $2$-torsion
	of $\symmetricFamily g 3 K$ is trivial, since the $2$-torsion precisely
	corresponds to the roots of $f(x)$ in $y^2 = f(x)$,
	\todo{make this more precise/understand better}
	we obtain that $\mono_{\symmetricFamily g 3 K} \subset \grouphypsmall g {\overline K}.$
	On the other hand,
	observe that over a dense open subscheme of $W \subset \moduliSpaceHyp g K$,
	the map $\phi_3|_{\phi_3^{-1}(W)}: \phi_3^{-1}(W) \rightarrow W$ as defined in~\eqref{equation:standard-family-and-symmetric-family}
	is a smooth covering whose geometric generic fiber has
	$(2g+2)!$ connected components corresponding
	to the $(2g+2)!$ orderings of the ramification points of the hyperelliptic
	curve.
	Let $Z := \phi_3^{-1}(W)$.
	Since, locally on the target, any smooth $X \rightarrow Y$ factors as
	$X \rightarrow \mathbb A^n_Y \rightarrow Y$, where
	the first map is \'etale map.
	Hence, after possibly replacing $W$ and $Z$ with smaller open subschemes,
	we can assume the map $\phi_3$ factors as $Z \xrightarrow \alpha \mathbb A^n_W \rightarrow W$.
	Since $\phi_3$ has $(2g + 2)!$ components in its geometric generic fiber,
	so does $\alpha$, and so after possibly further shrinking $W$,
	we can assume that $\alpha$ is an \'etale cover of degree
	$(2g+2)!$.
	Since the map $\pi_1(\mathbb A^n_W) \rightarrow \pi_1(W)$ is surjective
	as the map of schemes is dominant, it follows that the index
	$[\mono_W : \mono_Z] \leq (2g+2)!$.
	Of course, $\mono_W = \mono_{\symmetricFamily g 3 K}$ since
	it is $W \subset \symmetricFamily g 3 K$ is a dense open subscheme.
	Since we have shown above that $\mono_W \subset \Gamma_2$,
	we have both
	$[\mono_W :\mono_Z] \geq [\Gamma_2 : \grouphyp g {\overline K}] = (2g+2)!$
	and $[\mono_W : \mono_Z] \leq (2g+2)!$.
	It follows that $\mono_W = \Gamma_2$.

	To conclude, we use the geometric monodromy of the symmetric
	family to deduce the geometric monodromy of the standard family.
	Recall the map $\sigma_3: \symmetricFamily g 3 {\overline K} \rightarrow \standardFamily g 3 {\overline K}$.
	Note that this is a generically \'etale degree $(2g+1)!$ cover.
	To conclude the proof, we only need verify that the image of the 
	action on the two torsion of $\standardFamily g 3 {\overline K}$
	is given by the group 
	$S_{2g+1} \subset \Sp_{2g}(\mathbb Z/2 \mathbb Z)$,
	acting on the
	$2g+1$ ramification
	points of the family $\standardFamily g 3 {\overline K}$ (and the $2g+2$nd point
	is fixed at $\infty$).
	There are elementary algebraic ways to see this, but perhaps
	the simplest way is to use the equivalence of the \'etale
	fundamental group and the topological fundamental group,
	and then use the Lefschetz principle to replace $\overline K$
	by $\mathbb C$.
	Then, for any basepoint $[C] \in\standardFamily g 3 {\mathbb C}$,
one can find a loop in $\standardFamily g 3 {\mathbb C}$,
which permutes the ramification points of $[C]$ by an arbitrary way.
Therefore, the deck transformations of $\sigma_3$ are indeed identified
with $S_{2g+1}$, and since they act by permuting the $2g+1$ 
ramification points, the mod-$2$ image of this representation indeed
agrees with the reduction of $\grouphypsmall g {\mathbb C}$.
Since the mod $2$ reductions of $\grouphypsmall g {\mathbb C}$
and $\mono_{\standardFamily g 3 {\mathbb C}}$ agree and
both contain $\Gamma_2$, it follows that they are equal as subgroups
of $\Sp_{2g}(\widehat{\mathbb Z})$.
	\end{proof}


\subsection*{This obselete section contains the old version of Lemma 3.2, and its incorrect proof. These can be deleted. -James}
\begin{lemma*}
%Consider $S_{2g+1}$ and $S_{2g+2}$ as subgroups of $\Sp_{2g}(\bz / 2 \bz)$. Let $\hyphat_{2g+1}$ and $\hyphat_{2g+2}$ be their preimages in $\GSp_{2g}(\zh)$. These are the Galois images of the standard families of hyperelliptic curves. By Theorem~\ref{theorem:small-ab},
Let $g \geq 2$ and let $H \subset \GSp_{2g}(\widehat{\mathbb Z})$ be a subgroup.
%, and for a prime number $\ell$, let $H_\ell$ denote the $\ell$-adic reduction of $H$. 
Suppose that $H_2 \supset \hypaddict_{2g+i}$ for some $i \in \{1,2\}$ and that
$H(\ell) \supset \Sp_{2g}(\bz/\ell \bz)$ for all odd primes $\ell$.
%If $g = 1$, further assume that $H(9) \supset \Sp_{2}(\bz/9 \mathbb Z)$
Then
$[\hyphat_{2g+i} : [H,H]] = 2$, and in particular, $[\hyphat_{2g+i} : [\grouphypboth g i K,\grouphypboth g i K]] = 2$. \todo{The hypothesis $H_2 = \hypaddict_{2g+i}$ is needed. Otherwise, $H$ could equal all of $\GSp_{2g}(\zh)$, and the result fails. I leave this only as a comment, in case its a controversial change. -James}
%	\begin{itemize}
%		\item $[\hyphat_{2g+1}, \hyphat_{2g+1}]$ has index 2 in $\hyphat_{2g+1} \cap \Sp_{2g}(\zh)$, and 
%		\item $[\hyphat_{2g+2}, \hyphat_{2g+2}]$ has index 2 in $\hyphat_{2g+2} \cap \Sp_{2g}(\zh)$. 
%	\end{itemize}
\end{lemma*} 
%\begin{remark}
%Upon reading the statement of Corollary~\ref{theorem:r=2}, one can immediately proceed to read the proofs of Theorems~\ref{mainbldg} and~\ref{exemplinongratia}, which are detailed in Sections~\ref{section:proof-of-mainbldg} and~\ref{verified}, respectively.
%\todo{this remark should certianly be elsewhere, unless this
%lemma is placed in the group theory section}
%\end{remark}
\begin{proof} 
	Fix $i \in \{1, 2\}$. Since we are only interested in the commutator subgroup of $H$, it suffices to consider the case where $H \subset \Sp_{2g}(\widehat{\mathbb Z})$.
	Note that $H_\ell = \Sp_{2g}(\mathbb Z_\ell)$
	by \cite[Theorem 1]{landesman-swaminathan-tao-xu:lifting-symplectic-group}.
%	in the case $g \geq 2$ and 
%	~\cite[Lemma 3, Section IV.3.4]{serre1989abelian}
%	in the case $g = 1$. 
	\mbox{Then, by~\cite[Proposition 2.5]{landesman-swaminathan-tao-xu:rational-families}, we have}
	\begin{equation*}
		H = H_2 \times \prod_{\ell \ge 3} \Sp_{2g}(\bz_\ell) \supset \hypaddict_{2g+i} \times \prod_{\ell \ge 3} \Sp_{2g}(\bz_\ell).
	\end{equation*}
	Since our ``closed'' commutators respect infinite products, we have that 
    \begin{align*} 
    	[H, H] %&= [\hypaddict_{2g+i}, \hypaddict_{2g+i}] \times \prod_{\ell \ge 3} [\Sp_{2g}(\bz_\ell), \Sp_{2g}(\bz_\ell)] \\
        &\supset [\hypaddict_{2g+i}, \hypaddict_{2g+i}] \times \prod_{\ell \ge 3} [\Sp_{2g}(\bz_\ell), \Sp_{2g}(\bz_\ell)], 
    \end{align*} 
    By Theorem~\ref{theorem:small-ab} together with the fact that $[S_{2g+i}, S_{2g+i}] = A_{2g+i}$, the alternating group, we have that $[\hypaddict_{2g+i}, \hypaddict_{2g+i}] = \fy{2^\infty}{2}^{-1}(A_{2g+i})$. Also, $[\Sp_{2g}(\bz_\ell), \Sp_{2g}(\bz_\ell)] = \Sp_{2g}(\ZZ_\ell)$ by~\cite[Proposition 3(a)]{landesman-swaminathan-tao-xu:lifting-symplectic-group} when $g \geq 2$. 
   % by~\cite[Lemma A.1]{zywina2010elliptic} for $g = 1, \ell > 3$, and by direct computation for $g = 1, \ell = 3$.
   % \todo{question for david: is there a better reference for the $g = 1$ case?}
We deduce that
\begin{align*}
[H,H] &\supset \fy{2^\infty}{2}^{-1}(A_{2g+i}) \times \prod_{\ell \ge 3} \Sp_{2g}(\bz_\ell) = (\GSp_{2g}(\zh) \to \Sp_{2g}(\bz / 2 \bz))^{-1}(A_{2g+i}).
\end{align*}
In fact, the containment above is an equality, because
$\left[ H,H \right](2) = \left[ S_{2g+i}, S_{2g+i} \right] = A_{2g+i}$.
The desired result then follows from the fact that $[S_{2g+i} : A_{2g+i}] = 2$. 
\end{proof}

\subsection*{Obselete: Monodromy of Hyperelliptic Families // The improved version of `embedding the symmetric group' makes most of this section unnecessary. -James} \label{imtiredoflabellingthesethings}
% LOL

In this section, we compute the monodromy of the families $\standardFamily g i K \to \standardTarget g i K$ 
in Lemma~\ref{lemma:geometric-monodromy-of-standard-family}
and
of the moduli stack $\hyperelliptic g K$ of genus-$g$ hyperelliptic curves over $K$
in Lemma~\ref{lemma:moduli-stack-monodromy}, with its universal family. For concision, we shall refer to each family by its base, and omit writing the total space. 

Our strategy for computing the monodromy of $\hyperelliptic g K$ and $\standardTarget g i K$ is as follows: We first show that $\standardTarget g 2 K$ has the same monodromy
as $\hyperelliptic g K$.
We do this by demonstrating that the map $\standardTarget g 2 K \rightarrow \hyperelliptic g K$ has geometrically connected generic fiber
in Lemma~\ref{lemma:connected-generic-fiber}.
We then compute the monodromy of the families $\standardTarget g i K$ in Lemma~\ref{lemma:geometric-monodromy-of-standard-family},
and we finally deduce the monodromy of $\hyperelliptic g K$ in Lemma~\ref{lemma:moduli-stack-monodromy}.

The family $\standardTarget g 2 K$ was constructed by using an affine space to parameterize the coefficients of a polynomial $f(x)$ of degree $2g+2$, and hence a hyperelliptic curve $y^2 = f(x)$ given in Weierstrass form. In order to show $\standardTarget g 2 K$ and $\hyperelliptic g K$ have the same monodromy,
we need to introduce an auxiliary family, mapping to $\standardTarget g 2 K$, where we instead label the roots of $f(x)$.

\begin{definition}
	\label{definition:symmetric-families}
	Let $\symmetricTarget g 2 K := \mathbb A^{2g+2}_{[b_0, \ldots, b_{2g+1}]}$
and define the family $\symmetricFamily g 2 K \rightarrow \symmetricTarget g 2 K$ of genus $g$ over $K$
to be the smooth projective completion of the family of affine curves given by 
\[
%\symmetricFamily g 1 K &:= V(y^2 - (x - b_{2g}) \cdots (x -b_0)) \subset \mathbb A^{2g+1}_{[b_0, \ldots, b_{2g}]} \times\mathbb A^2_{[x,y]}, \\
	V(y^2 - (x - b_{2g+1}) \cdots (x -b_0)) \hookrightarrow \symmetricTarget g 2 K \times \mathbb A^2_{[x,y]} \rightarrow \symmetricTarget g 2 K. \\
%\symmetricFamily g 3 K &:= V\left( y^2 - (x - b_{2g}) \cdots (x-b_1) \cdot \left( x + \sum_{j=1}^{2g} b_j  \right)\right) \subset \mathbb A^{2g+2}_{[b_1, \ldots, b_{2g}]} \times \mathbb A^2_{[x,y]}, \\
%	\widetilde{\symmetricFamily g 4 K} & \defeq V\left (y^2 - (x - b_{2g+1}) \cdots (x -b_1) \cdot \left( x + \sum_{j=1}^{2g+1} b_j  \right)\right) \subset \mathbb A^{2g+1}_{[b_1, \ldots, b_{2g+1}]} \times \mathbb A^2_{[x,y]}.
\]
\end{definition}

By construction, we have the following fibered diagram: 
\[
\begin{tikzcd}
\symmetricFamily{g}{2}{K} \ar{r} \ar{d} & \standardFamily{g}{2}{K} \ar{d} \\
\symmetricTarget{g}{2}{K} \ar{r}{\sigma_2} & \standardTarget{g}{2}{K}
\end{tikzcd}
\]
where $\sigma_2$ is a map between affine spaces, defined by the elementary symmetric polynomials in $b_0, \ldots, b_{2g+1}$. Furthermore, the families of hyperelliptic curves defined over $\symmetricFamily g 2 K$ and $\standardFamily g 2 K$ are represented by maps from these spaces to $\hyperelliptic g K$. We hence obtain maps to the associated coarse moduli space $\moduliSpaceHyp g K$, as shown in this commutative diagram: 
\begin{equation}
	\label{equation:standard-family-and-symmetric-family}
	\begin{tikzcd}
		\symmetricTarget g 2 K \ar {rr}{\sigma_2} \ar{rd}{\Phi_2} \ar[swap]{rdd}{\phi_2} && \standardTarget g 2 K \ar {ldd}{\psi_2} \ar[swap]{ld}{\Psi_2} \\
		& \hyperelliptic g K \ar{d} & \\
		 & \moduliSpaceHyp g K. & 
	 \end{tikzcd}\end{equation}

\begin{lemma}
	\label{lemma:connected-generic-fiber}
	The geometric generic fibers of the maps $\Psi_2$ and $\psi_2$ defined in~\eqref{equation:standard-family-and-symmetric-family} are connected.
%	Furthermore, there is an open subcheme of $W \subset \standardTarget g i K$ so that the map $\sigma_i|_{\sigma_i^{-1}(W)}$ is a covering map of degree $(2g+2)!$.
\end{lemma}
\begin{proof}
%	We prove this only in the case that $i = 2$ for the sake of brevity, because the case $i = 4$ is completely analogous.
	In order to show that the geometric generic fiber of $\Psi_2$
	is connected, it suffices to show that the geometric generic fiber of $\psi_2$ is connected.
	In turn, using \cite[Book IV, Th\'eor\`eme 9.7.7(ii)]{EGA}
	(constructibility of the geometrically connected locus)
	it suffices to show that for a general closed point of
	$\moduliSpaceHyp g K$, the geometric fiber \mbox{of $\psi_2$
	is connected.}

	We now describe the fiber of $\phi_2$ over a closed
	geometric point $\overline{[C]} \in \moduliSpaceHyp g K$ corresponding to a curve $C/\ol{K}$, 
	showing that this fiber has $(2g+2)!$ connected components.
	Suppose that $C$ is isomorphic over $\overline K$
	to the curve cut out by the equation $y^2 = \prod_{j=1}^{2g+2} (x-b_j)$.
	Then, $\PGL_2(\ol{K})$ acts on the $b_j$ as points in $\mathbb{P}_{\ol{K}}^1$
	and the symmetric group $S_{2g+2}$ acts by permuting the $b_j$.
	All curves in the fiber of $\phi_2$ are of the form
	$y^2 = \prod_{j=1}^{2g+2} \left( x-\eta (\tau (b_j)) \right)$
	for $\eta \in S_{2g+2}, \tau \in \PGL_2(\ol{K})$.
	If the $b_j$ are chosen generally, there will be no nontrivial
	$(\tau,\eta)$ so that $\left\{ \tau(b_1), \ldots, \tau(b_{2g+2}) \right\}  = \left\{ \eta(b_1), \ldots, \eta(b_{2g+2}) \right\}$.
	Hence, there will be $(2g+2)!$ connected components of the fiber,
	corresponding to the $\PGL_2(\ol{K})$ orbits of the $(2g+2)!$ elements
	$\left\{ \eta(b_1), \ldots, \eta(b_{2g+2}) \right\}$ as $\eta$ ranges over elements
	of $S_{2g+2}$.
	Strictly speaking, by the way we constructed our families,
	we do not allow any of the roots of the hyperelliptic
	curves to lie at $\infty \in \mathbb{P}_{\ol{K}}^1$. 
	So, each of the
	connected components of the fiber is
	parameterized by an open subscheme of $\PGL_2(\ol{K})$, which is still connected.

	Now, let $V_1, \ldots, V_{(2g+2)!}$ be the connected components of
	such a fiber of $\phi_2$. Observe that $V_1$
	maps dominantly under $\sigma_2$ 
	onto the corresponding fiber of $\psi_2$.
		It follows that a general
	closed fiber of $\phi_2$ is geometrically connected,
	as we wanted to show.
\end{proof}

We next use Lemma~\eqref{lemma:connected-generic-fiber} to verify that \emph{any} family of hyperelliptic curves has monodromy contained in $\grouphyp g K$.

\begin{lemma}
	\label{lemma:monodromy-containment}
	Let $g \geq 2$. Any family $A \rightarrow U$ of Jacobians of hyperelliptic curves satisfies
$\mono_{A} \subset \grouphypboth g {2} K$.
Furthermore, for $i \in \{1, 2, 3, 4\}$, we have that
$\mono_{\standardFamily g i K} \subset \grouphypboth g {(i \bmod 2)} K$.
\end{lemma}
\begin{proof}
%	First, we reduce the question of verifying the 
%	the statement for arbitrary curves
%	to verifying $\mono_{\standardFamily g 2 K} \subset
%	\grouphyp g K$.
	To prove the statement for arbitrary
	families, it suffices to prove that
	$\mono_{\hyperelliptic g K} \subset \grouphyp g K$.
	However, since the geometric generic fiber of the map $\Psi_2$ from 
	~\eqref{equation:standard-family-and-symmetric-family} is connected by Lemma~\ref{lemma:connected-generic-fiber} and
	\cite[Proposition 5.2]{landesman-swaminathan-tao-xu:rational-families} (which says that in nice circumstances, a map from a scheme
	to a stack with connected geometric generic fiber induces 
	a surjective map of fundamental groups)
	it follows
	that $\mono_{\hyperelliptic g K} = \mono_{\standardFamily g 2 K}$. Thus, to complete the proof, we only need show that
$\mono_{\standardFamily g i K} \subset \grouphypboth g {(i \bmod 2)} K$ for each $i \in \{1, 2, 3, 4\}$.
	The condition that $\mono_{\standardFamily g i K} \subset \grouphyp g K$ is equivalent to the following
	two conditions: 
    \begin{enumerate}
    \item[(a)] $\mono_{\standardFamily g i K} \subset \on{mult}^{-1}(\chi(K))$,
	\item[(b)] $\mono_{\standardFamily g i K}(2) \subset S_{2g+(i \bmod 2)} \subset \Sp_{2g+2}(\mathbb Z/2\mathbb Z)$.
    \end{enumerate}
	Condition (a) holds by Remark~\ref{remark:det-rho-is-chi}.
	As for Condition (b), note that by the version of Hilbert's Irreducibility Theorem \todo{not needed anymore, but it would be important that the base is rational to apply this} stated in~\cite[Theorem 1.2]{zywina2010hilbert}, most $K$-valued points of $U$ will have the same mod-$2$ monodromy
	as that of the family. By Section~\ref{symbed}, all $K$-valued points have mod-$2$ monodromy contained in $S_{2g+(i \bmod 2)}$, so Condition (b) holds.
\end{proof}

We now show that the containment in Lemma~\ref{lemma:monodromy-containment} is an equality for the families $\standardFamily g i K$.
\begin{lemma}
	\label{lemma:geometric-monodromy-of-standard-family}
	Let $g \geq 2$. For $i \in \{1, 2, 3, 4\}$, we have that $\mono_{\standardFamily g i K}= \grouphypboth g {(i \bmod 2)} {K}$ and for any algebraically closed field $L$ which is a subfield of $\mathbb C$, we have $\mono_{\standardFamily g i L}= \grouphypboth g {(i \bmod 2)} {L}$.
\end{lemma}
\begin{proof}
We reduce the first statement for number fields $K$ to the second statement
for algebraically closed fields in the case $L = \overline K$.
Observe that for each $i \in \{1, 2, 3, 4\}$ we have a map of short exact sequences
\begin{equation}
	\label{equation:}
	\begin{tikzcd}
		0 \ar {r}  &  \mono_{\standardFamily g i {\overline K}} \ar {r}\ar{d} & \mono_{\standardFamily g i K} \ar {r}{\mult}\ar{d} & \chi(K) \ar {r}\ar[equal]{d} & 0 \\
			0 \ar {r} &  \grouphypboth g {(i \bmod 2)} {\overline K} \ar {r} & \grouphypboth g {(i \bmod 2)} K \ar[swap]{r}{\mult} & \chi(K) \ar {r} & 0.
	\end{tikzcd}\end{equation}
%The first vertical map comes from the assumption that we know the statement holds over $\mathbb C$, and second vertical map
%is an inclusion by Lemma~\ref{lemma:monodromy-containment}.
Since the maps in~\eqref{equation:} are all inclusions or the $\mult$ map, the diagram commutes. By the Five Lemma, the second vertical map is an isomorphism if and only if the first one is.

So, we only need prove the statement regarding algebraically closed
fields $L$ of characteristic $0$ contained in $\mathbb C$.
We now reduce to the case $L = \mathbb C$.
Because $\mathbb C$ is a base extension of $L$, we have an injection
$\pi_1(\standardTarget g i {\mathbb C}) \rightarrow \pi_1(\standardTarget g i {\overline K})$, implying
$\mono_{\standardTarget g i {\mathbb C}} \subset \mono_{\standardTarget g i {\overline K}}.$
However, if we assume $\pi_1(\standardTarget g i {\mathbb C}) =\grouphypboth g i {\mathbb C}$,
(which is the case $L = \mathbb C$,) then
by Lemma~\ref{lemma:monodromy-containment},
$\mono_{\standardTarget g i {\overline K}} \subset \grouphypboth g i {\overline K} = \grouphypboth g i {\mathbb C} = \mono_{\standardTarget g i {\mathbb C}}$. Hence, $\mono_{\standardTarget g i {\overline K}} = \mono_{\standardTarget g i {\mathbb C}}.$

Therefore, it suffices to show $\mono_{\standardFamily g i {\mathbb C}} = \grouphypboth g {(i \bmod 2)} {\mathbb C}$.
This statement for $i = 1,2$ follows from~\cite[Th\'eor\`eme 1]{acampo:tresses-monodromie-et-le-groupe-symplectique}
	(since the \'etale fundamental group is the profinite completion of the
	topological fundamental group).
To complete the proof we
only need show that when $i = 3, 4$
we have $\mono_{\standardFamily g i {\mathbb C}} = \mono_{\standardFamily g {i-2} {\mathbb C}}$.
However, this claim follows because the family
$\standardTarget g {i-2} {\mathbb C}$ topologically deformation retracts
onto $\standardTarget g i {\mathbb C}$, and hence has the
same monodromy group.
We now construct this deformation retract explicitly.
A hyperelliptic curve
$y^2 = f(x)$ over $\mathbb C$ can be written as
$C := V\left(y^2 - \prod_{j=1}^{2g+i-2} \left( x-b_j \right) \right)$.
Then, the deformation retract of $\standardTarget g {i-2} {\mathbb C}$ onto $\standardTarget g i {\mathbb C}$ is given by the map
\begin{align*}
 \standardTarget g {i-2} {\mathbb C} \times [0,1] 
& \rightarrow \standardTarget g {i-2} {\mathbb C}\\
%	\left(  V\left(y^2 - \prod_{j=1}^{2g+i-2} \left( x-b_j \right) \right), t\right)
% & \mapsto V \left( y^2 - \prod_{j=1}^{2g+i-2}\left( x- b_j + \frac{t}{2g+i-2} \cdot \sum_{i=0}^{2g+i-2} b_j \right) \right)
% \left((b_1, \ldots, b_{2g+i-2}), t\right)
 %& \mapsto (b_1 + \frac{t}{2g+i - 2}, \ldots, b_{2g+i-2} + \frac{t}{2g+i - 2})
\left((b_1, \ldots, b_{2g+i-2}), t\right)
& \mapsto (b_1 + \frac{t}{2g+i - 2}, \ldots, b_{2g+i-2} + \frac{t}{2g+i - 2})
\end{align*}
Therefore, $\pi_1(\standardTarget g {i-2}{\mathbb C}) \cong \pi_1(\standardTarget g i {\mathbb C})$, and the two families also have the same monodromy.
\end{proof}

We can now easily deduce the monodromy of the moduli stack of hyperelliptic
curves from the previous lemmas:
\begin{lemma}
	\label{lemma:moduli-stack-monodromy}
	Let $g \geq 2$. We have that $\mono_{\hyperelliptic g K} = \mono_{\moduliSpaceHyp g K} = \grouphyp g {K}$.
\end{lemma}
\begin{proof}
By Lemma~\ref{lemma:connected-generic-fiber}, each of the maps $\Psi_2 \colon \standardTarget g 2 K \rightarrow \hyperelliptic g K$ and $\psi_2 \colon \standardTarget g 2 K \rightarrow \moduliSpaceHyp g K$ has geometrically connected generic fiber.
	By~\cite[Proposition 5.2]{landesman-swaminathan-tao-xu:rational-families},
	the induced maps $\pi_1(\standardTarget g 2 K) \rightarrow \pi_1(\hyperelliptic g K)$ and $\pi_1(\standardTarget g 2 K) \rightarrow \pi_1(\moduliSpaceHyp g K)$ are both surjective. We then have that \mbox{$\mono_{\hyperelliptic g K} = \mono_{\moduliSpaceHyp g K} = \mono_{\standardFamily g 2 K}$,}
so the desired result follows from Lemma~\ref{lemma:geometric-monodromy-of-standard-family}.
%	Recall that
%	$\grouphyp g {\overline K}$ is precisely
%	the set of matrices $M \in \Sp_{2g}(\widehat{\mathbb Z})$ with
%	$M \bmod 2 \in S_{2g +2} \subset \Sp_{2g}(\mathbb Z/2 \mathbb Z)$,
%	via the inclusion given in Lemma~\ref{theorem:include-s}.
%
%
%	By \todo{cite appropriate version of goursat's lemma}
%	to conclude the proof, it suffices to show that the $\bmod 2$ image
%	of the geometric monodromy of $\hyperelliptic g K$ is
%	precisely $S_{2g+2} \subset \Sp_{2g}(\widehat{\mathbb Z})$.
\end{proof}

