\section{Proof of Theorem~\ref{mainbldg}}
\label{section:proof-of-mainbldg}

%\todo{Aaron: this whole section is not correctly stated if we take $g = 1$, since $\Sp_2(\mathbb Z/2) = S_3$, so it is not possible to have $\bmod 2$ monodromy equal to $S_4$.}
In this section, we prove the first main result of this paper, namely Theorem~\ref{mainbldg}. We begin in Section~\ref{mygawdsadiomaneisamazing} with a description of the relevant background material on Galois representations of PPAVs. Then, in Section~\ref{prelimswine}, 
we prove a group-theoretic Lemma, useful for determining $\delta_K$.
In Section~\ref{symbed}, we describe the particular manner in which we embed $S_{2g+2}$ as a subgroup of $\Sp_{2g}(\ZZ/2 \ZZ)$.
In Section~\ref{subsection:monodromy-of-families}
we determine the monodromy groups of the four families of hyperelliptic curves introduced in Definition~\ref{definition:standard-families} and the monodromy of the universal family over the moduli stack of hyperelliptic curves. Finally, in Section~\ref{32isnotagoodscoreline}, we complete the proof of Theorem~\ref{mainbldg}.

\subsection{Background}\label{mygawdsadiomaneisamazing}

Let $K$ be a number field, let $r \geq 0$ be an integer, and let $U \subset \mathbb{P}_K^r$ be an open subscheme. For an integer $g \geq 0$, let $A$ be a family of $g$-dimensional PPAVs over $U$, by which we mean that $A$ is an abelian scheme over $U$, meaning that $A \rightarrow U$ is a proper smooth group scheme with geometrically connected fibers of dimension $g$, and $A$ is equipped with a principal polarization over $U$.  Because the base $U$ is rational, we call $A \to U$ a \emph{rational family}. By construction, the fiber $A_u$ of the morphism $A \to U$ over any $K$-valued point $u \in U(K)$ is a $g$-dimensional PPAV over $K$.

%Let $\ol{\eta}$ be a geometric generic point of $U$, and
Recall that the action of the \'{e}tale fundamental group $\pi_1(U)$ on the %$m$-
torsion points of a chosen geometric generic fiber of $A \to U$ gives rise to a continuous linear representation whose image is constrained by the Weil pairing to lie in the general symplectic group $\GSp_{2g}(\wh{\ZZ})$. We denote the resulting \emph{adelic representation} by 
%\emph{mod-$m$ representation} by
%\begin{equation}\label{atoll}
%\rho_{A,m} \colon \pi_1(U, \ol{\eta}) \to \GSp_{2g}(\ZZ/m \ZZ).
%\end{equation}
%We can take the inverse limit of the mod-$m$ representations to obtain an \emph{adelic representation}
\begin{equation}\label{thisisthepartofme}
	\rho_A \colon \pi_1(U) \to \GSp_{2g}(\wh{\ZZ}).
\end{equation}
%Taking the $\ell$-adic reduction of the adelic representation (or equivalently the inverse limit over all mod-$\ell^k$ representations) yields the \emph{$\ell$-adic representation}
%\begin{equation}\label{dontforgetme}
%	\rho_{A,\ell} \colon \pi_1(U, \ol{\eta}) \to \GSp_{2g}(\ZZ_\ell).
%\end{equation}


We now define the monodromy groups associated to $\rho_A$. We call the image of $\rho_A \colon \pi_1(U) \to \GSp_{2g}(\wh{\ZZ})$ the {\it monodromy} of the family $A \to U$, and we denote it by $\mono_A$. We write $\mono_A(m)$ for the mod-$m$ reductions and $\mono_{A,\ell}$ for the $\ell$-adic reductions of the above-defined monodromy groups.

\begin{remark}
Let $u \in U(K)$ be a $K$-valued point. Precomposing the adelic representation with the induced map $\pi_1(u) \to \pi_1(U)$ gives a representation $\pi_1(u) \to \GSp_{2g}(\wh{\ZZ})$ whose image we denote by $\mono_{A_u}$. 
Because $\pi_1(u) \simeq G_K$,
the representation $\rho_{A_u}$ obtained by restricting $\rho_A$ to $A_u$
is the same as the adelic representation $\rho_{A_u}$ discussed in Section~\ref{subsection:intro-background}.
\end{remark}

%\begin{remark}
%For a commutative ring $R$, recall from the definition of the general symplectic group that we have a multiplier map $\on{mult} \colon \GSp_{2g}(R) \to R^\times$. Let $\chi_m$ be the mod-$m$ cyclotomic character, and let $\chi$ be the cyclotomic character. If $U = \spec K$, it follows from $G_K$-invariance of the Weil pairing that $\chi_m = \mult \circ \rho_{A,m}$ and $\chi = \mult \circ \rho_{A}$.
%\end{remark}

%To recall some necessary notation, let $K$ be a number field, let $r \geq 0$ be an integer, and let $U \subset \mathbb{P}_K^r$ be an open subscheme.
%We say $A \rightarrow U$ is a family of $g$-dimensional PPAVs
%if
%$A$ is a group scheme over $U$ with a principal polarization so that
%$A \rightarrow U$ is proper, smooth, with geometrically connected fibers
%of dimension $g$.
%We let $\rho_A \colon \pi_1(U) \rightarrow \GSp_{2g}(\widehat{\mathbb Z})$
%denote the adelic Galois representation and $\rho_{A,m} \colon \pi_1(U) \rightarrow \GSp_{2g}(\widehat{\mathbb Z})$ its mod-$m$ reduction.
%We define the {\it monodromy} of the family $A \rightarrow U$, denoted $\mono_A$ as the image of $\rho_A$ in $\GSp_{2g}(\widehat{\mathbb Z})$ and
%the {\it geometric monodromy}, denoted by $\mono_A^{\on{geom}}$, 
%as the image of $\rho_{A_{\overline K}}\colon \pi_1(U_{\overline K}) \rightarrow \GSp_{2g}(\widehat{\mathbb Z})$.
%Since the cyclotomic character is trivial on $G_{\ol{K}}$, it follows that $\mono_A^{\on{geom}}$ is actually a subgroup of $\Sp_{2g}(\wh{\ZZ})$. 
%We next recall the relationship between the Galois
%representation and the cyclotomic character:
\begin{remark}
	\label{remark:det-rho-is-chi}
For a commutative ring $R$, recall from the definition of the general symplectic group that we have a multiplier map $\on{mult} \colon \GSp_{2g}(R) \to R^\times$. If $\chi$ denotes the cyclotomic character, then for a PPAV $A$ it follows from $G_K$-invariance of the Weil pairing that $\chi = \mult \circ \rho_{A}$.
More generally, if $A \rightarrow U$ is a family
of PPAVs with $U$ normal and integral, and if $\phi$ denotes the map $\pi_1(U) \rightarrow \pi_1(\spec K)$ induced by the structure map $U \to \spec K$, then we have that
$\chi \circ \phi = \mult \circ \rho_A$%, which holds because it holds for the generic fiber $A_\eta \rightarrow \spec K(\eta)$, and the map $\pi_1(\eta) \rightarrow \pi_1(U)$ is surjective.
\end{remark}

\subsection{Computing $\delta_K$}
\label{prelimswine}

In this section, we prove Lemma~\ref{theorem:r=2},
which is used to compute the value of $\delta_K$ in the
proof of Theorem~\ref{mainbldg}, given in Section~\ref{subsection:proof-of-first-claim}.
In order to state Lemma~\ref{theorem:r=2},
we need the following definition, in which
we introduce notation used throughout the paper to denote various lifts of $S_{2g+i}$:

\begin{definition}\label{grpnotes}
	For $i \in \left\{ 1,2 \right\}$,
	we define 
	%For $m \in \ZZ_{> 0} \cup \{\infty\}$ and $g \geq 2$, denote by $\gstraw{2^m} \colon \GSp_{2g}(\wh{\ZZ}) \to \GSp_{2g}(\ZZ/2^m \ZZ)$ and $\straw{2^m} \colon \Sp_{2g}(\wh{\ZZ}) \to \Sp_{2g}(\ZZ/2^m \ZZ)$ the mod-$2^m$ reduction maps (or the 2-adic projection, if $m = \infty$). For $i \in \{1, 2\}$, we shall use the following notation:
    % Let me explain this change: \straw and \gstraw are complicated notations that are barely used in the paper. I commented them out, and only needed to change this definition, and the proof of the following lemma. -James 
\begin{equation*}
 \hyphat_{2g+i} \defeq (\Sp_{2g}(\zh) \to \Sp_{2g}(\bz / 2 \bz))^{-1}(S_{2g+i}) \qquad \text{and} \qquad \hypaddict_{2g+i} \defeq \fy{2^\infty}{2}^{-1}(S_{2g+i}).
\end{equation*}
\end{definition}
The next lemma applies Theorem~\ref{theorem:small-ab} to determine how large the commutator subgroup of $\grouphypboth{g}{i}{K}$ is as a subgroup of $\hyphat_{2g+i}$:
\begin{lemma} \label{theorem:r=2}
	Let $g \ge 2$, let $i \in \left\{ 1,2 \right\}$
	and let $H \subset \GSp_{2g}(\zh)$ be a closed subgroup. Suppose that 
\begin{itemize} 
\item $H_2 = \gify{2^\infty}{2}^{-1}(S_{2g+i})$, and 
\item $H(\ell) \supset \Sp_{2g}(\bz / \ell \bz)$ for $\ell \ge 3$.
\end{itemize} 
Then 
\[
[H, H] = \fy{2^\infty}{2}^{-1}(A_{2g+i}) \times \prod_{\ell \ge 3} \Sp_{2g}(\bz_\ell),
\]
where $A_{2g+i}$ denotes the alternating group on $2g + i$ letters.
\end{lemma} 
\begin{proof} 
By Theorem~\ref{theorem:small-ab}, 
\begin{align*} 
[H, H]_2 = [H_2, H_2] = \fy{2^\infty}{2}^{-1}(A_{2g+i}). 
\end{align*} 
Also, note that, for $\ell \ge 3$, 
\begin{align*}
[H, H](\ell) &= [H(\ell), H(\ell)] \\
&\supset [\Sp_{2g}(\bz / \ell \bz), \Sp_{2g} (\bz / \ell \bz)] \\
&= \Sp_{2g}(\bz / \ell \bz), 
\end{align*} 
the last equality following from \cite[3.3.6]{omeara1978symplectic}. Now, applying \cite[Proposition 2.5]{landesman-swaminathan-tao-xu:rational-families} to $[H, H] \subset \Sp_{2g}(\zh)$ gives the result. 
\end{proof} 
\begin{corollary} \label{theorem:r=2:special}
For $H$ as in Lemma~\ref{theorem:r=2}, we have $[\hyphat_{2g+i} : [H, H]] = 2$. In particular, $[\hyphat_{2g+i} : [\grouphypboth g i K, \grouphypboth g i K]] = 2$. 
\end{corollary} 

\subsection{Embedding the Symmetric Group, Take 2} \label{symbed}

In Section~\ref{take1} we constructed the well-known embedding $S_{2g+2} \hookrightarrow \Sp_{2g}(\bz / 2 \bz)$. Beginning with
\begin{align*}
	V &\simeq \FF_2^{2g+2} \\
	t &= (1, \dots, 1) \in V \\
	W &= t^{\perp}/\on{span}(t),
\end{align*}
we observed that the action of $S_{2g+2}$ on the basis vectors of $V$ descends to a symplectic action on $W$. Our goal in this section is to relate this embedding with the mod-2 Galois representation attached to a family of hyperelliptic curves, by proving the following result:
\begin{theorem} \label{monodromy-in-sym}
	Given a family $\mc{C} \to \mc{U}$ of hyperelliptic curves (where $\mc{U}$ is any stack), and a geometric generic point $\ol{\eta} \hookrightarrow \mc{U}$, the monodromy group $\rho_{2}(\pi_1(\mc{U}, \ol{\eta})) \subset \Sp_{2g}(\bz / 2 \bz)$ is in fact contained in $S_{2g+2} \subset \Sp_{2g}(\bz / 2 \bz)$. As a subgroup of $S_{2g+2}$, the monodromy group is given by the action of $\pi_1(\mc{U})$ on the Weierstrass points of $\mc{C}_{\ol{\eta}}$.
\end{theorem}
This has an immediate consequence for our standard families:
\begin{corollary} \label{sym-W}
	For $i \in \{1, 2, 3, 4\}$, we have $\mono_{\standardFamily g i K} \subset \grouphypboth g {(i \bmod 2)} K$.
\end{corollary}
\begin{proof}
	We have $\on{mult}(\mono_{\standardFamily g i K}) = \chi(K)$ as subgroups of $\zh^\times$, by Remark~\ref{remark:det-rho-is-chi}. Therefore, it suffices to show that $\mono_{\standardFamily g i K}(2) \subset S_{2g+i}$. By Theorem~\ref{monodromy-in-sym}, we need only check that the monodromy action on the Weierstrass points of $\standardFamily g i K \to \standardTarget g i K$ is contained in $S_{2g+i}$. The nontrivial cases $i = 1, 3$ follow by observing that, when the defining equation is $y^2 = f(x)$ with $\deg f(x) = 2g+1$, one Weierstrass point always lies over infinity, hence is fixed under monodromy.
\end{proof}

We prove Theorem~\ref{monodromy-in-sym} in three steps:
\begin{enumerate}[(1)]
	\item In subsection~\ref{sec:single}, we prove the statement when $\mc{U}$ is $\spec \ol{k}$. The key points are that the constructions are functorial in $\mc{C}$ and that the isomorphism with $\on{Jac}_{C/U}[2]$ follows from standard facts about divisors on hyperelliptic curves.
	\item In subsection~\ref{part2}, we prove the statement when $\mc{U}$ is a scheme by explicitly constructing the algebraic space of Weierstrass points over $\mc{U}$, the corresponding group space $t^\perp / \on{span}(t)$ over $\mc{U}$, and the map from the latter to $\on{Jac}_{C/U}[2]$. Step (1) implies that this map is an isomorphism.
	\item In subsection~\ref{part3}, we interpret the (functorial) constructions of step (2) as giving rise to corresponding objects and maps over the moduli space $\hyperell_g$ of hyperelliptic curves, corresponding to the universal family $\hyperellSource_g \to \hyperell_g$.
\end{enumerate}



\subsubsection{A single hyperelliptic curve} \label{sec:single}
Let $k$ be an algebraically closed field of characteristic zero, let $C$ be a hyperelliptic curve over $k$, and let $J$ be the Jacobian of $C$. The
set of Weierstrass points $\{P_1, \ldots, P_{2g+2}\}$ of $C$ is uniquely determined because $g \ge 2$. With this setup, define $V$
to be the free vector space over $\FF_2$ spanned by $P_1, \ldots, P_{2g+2}$, so that
\begin{align*}
	t &= P_1 + \cdots + P_{2g+2} \\
	t^\perp &= \on{span}_{\mathbb{F}_2}(P_i - P_j : i, j \in \{1, \ldots, 2g+2\}).
\end{align*}
The map
\[
\begin{tikzcd}[row sep = 0.15cm]
\on{span}_{\bz}(P_1, \ldots, P_{2g+2}) \ar{r}{\phi} & \on{Pic}_C \\
\sum_i a_i \cdot P_i  \ar[mapsto]{r} & \mc{O}_C\left( \sum_i a_i \cdot P_i \right),
\end{tikzcd}
\]
is such that $\phi(\on{span}_{\bz}(P_i - P_j)) \subset \on{Pic}_C^0 \simeq J$. Furthermore, it can be checked that
\begin{itemize}
	\item The resulting map $\on{span}_{\bz}(P_i - P_j) \to J$ annihilates $2\cdot (P_i - P_j)$ and $t$. Hence it descends to a map $W \defeq t^\perp / \on{span}(t) \to J[2]$ of $\FF_2$ vector spaces.
	\item This latter map is an isomorphism.
\end{itemize}
The second bullet point implies that the action of $\on{Aut}(C)$ on $J[2]$, \emph{a priori} contained in $\Sp_{2g}(\bz / 2 \bz)$, is in fact contained in the subgroup $S_{2g+2} \subset \Sp_{2g}(\bz / 2 \bz)$ which is determined by the vector space isomorphism $J[2] \simeq W$. For details, see \cite[Proposition 1.2.1(a)]{yelton2015thesis}.


\subsubsection{Schematic families of hyperelliptic curves} \label{part2}
Let ${C} \to {U}$ be a family of hyperelliptic curves of genus $g$, where $U$ is a scheme. Because all constructions in \ref{sec:single} were functorial, they can be carried out in families. Let us indicate how this is done.
\begin{enumerate}[itemsep=0.25cm]
	\item Let ${P}$ be the fixed point locus of the hyperelliptic involution. Then we have a diagram
	\[
	\begin{tikzcd}[column sep = 2 cm]
	{P} \ar[hookrightarrow]{r}{\text{closed emb.}} \ar[swap]{rd}{\text{\'etale}} & {C} \ar{d} \\
	& {U}
	\end{tikzcd}
	\]
	For any geometric point $u \hookrightarrow {U}$, the fiber ${P}_u$ consists of the Weierstrass points of ${C}_u$.
	\item Let ${G}$ be the group algebraic space over ${U}$ which represents the sheaf associated to the following presheaf on $\on{Sch}_{/{U}}$, in the \'etale topology:
	\[
	T \mapsto \on{span}_\bz (\on{Hom}_{{U}}(T, {P})).
	\]
	Representability follows by taking an \'etale cover of ${U}$ which trivializes ${P}$. For $u \ra {U}$ a geometric point, the fiber ${G}_u$ equals $\on{span}_\bz({P}_u)$.
	
	There is a section $t \colon {U} \to {G}$ which is first defined on a sufficiently fine \'etale cover ${U}' \to {U}$ for which $U' \times_U P$ is a trivial $(2g+2)$-cover of $U'$, by ``adding all the Weierstrass points,'' i.e.\ summing the $(2g+2)$ basis elements of
	\[
		\on{span}_\bz(\Hom_U(U', P)) \simeq \on{span}_\bz(\Hom_{U'}(U', U' \times_U P)).
	\]
	The section on $U'$ can then be descended to $U$.
	
	We can also define a group subspace ${G}^0 \hookrightarrow {G}$ via the sub-presheaf given by requiring that the coefficients of the $\bz$-linear combination sum to zero.
	\item Define a map $\Phi: {G} \to \on{Pic}_{{C}/{U}}$ of group spaces over ${U}$ as follows: given $f \in {G}(T)$, we may find an \'etale cover $\sigma: T' \to T$ for which $\sigma^* f = \sum_i f_i$, for some $f_i \in \Hom_{{U}}(T', {P})$. Each $f_i$ gives a section of the pulled-back family ${C}_{T'} \to T'$, whose image determines a relative effective Cartier divisor $D_i$. A standard descent argument shows that $\sum_i D_i$ descends to a divisor $D$ on ${C}_T$, which does not depend on the chosen \'etale cover $\sigma$. We may therefore define $\Phi(f) \defeq D$. This assignment is natural in $T$, so it gives a natural transformation of functors ${G} \to \on{Pic}_{{C}/{U}}$. The fiber $\Phi_u$ is the map $\phi$ defined in \ref{sec:single}. The map $\Phi$ restricts to a map ${G}^0 \to \on{Pic}_{{C}/{U}}^0 \simeq \on{Jac}_{{C}/{U}}$.
	\item Subsection \ref{sec:single} shows that $\Phi$ is surjective and allows us to describe the kernel as follows. First, $2 \cdot G^0$ maps to zero, so $\Phi$ descends to a map $G^0 / (2 \cdot G^0) \to \on{Jac}_{C/U}$. Second, the inclusion $G^0 \hookrightarrow G$ gives an injection $G^0 / (2 \cdot G^0) \hookrightarrow G / (2 \cdot G)$, and the image of $t \in G(U)$ in the quotient $G / (2\cdot G)$ in fact lies in $G^0 / (2 \cdot G^0)$ because $t$ is a sum of an even number of terms; we abuse notation by denoting the latter section with the same symbol $t$. This $t$ spans the kernel of the descended map $G^0 / (2 \cdot G^0) \to \on{Jac}_{C/U}$, so that
	\[
		G^0 / (2 \cdot G^0 + \on{span}(t)) \to \on{Jac}_{C/U}
	\]
	is an isomorphism of group stacks over $U$.
\end{enumerate}
This proves Theorem~\ref{monodromy-in-sym} when $\mc{U} := U$ is a scheme, because the action of $\pi_1(U)$ on the fiber of $G^0 / (2G^0 + \on{span}(t))$ over a chosen geometric generic point $\ol{\eta} \in U$ (as an $\mathbb{F}_2$-vector space) is obtained from the action of $\pi_1(U)$ on the fiber of $P_{\ol{\eta}}$ (as a set of size $(2g+2)$) via the procedure of Section~\ref{take1}.

\subsubsection{The universal family over $\hyperell_g$} \label{part3}

The constructions of subsection~\ref{part2} are functorial in the chosen family $C \to U$, and behave well under base change $V \to U$, so we obtain the analogous constructions over the moduli stack of hyperelliptic curves:
\[
\begin{tikzcd}[column sep = 0.7in]
\mc{P} \ar[hookrightarrow]{r}{\text{closed emb.}} \ar[swap]{rd}{\text{\'etale}} & \hyperellSource_g \ar{d} \\
& \hyperell_g
\end{tikzcd}
\quad \text{and} \quad
\begin{tikzcd}[column sep = 0.7in]
\mc{G}^0 \ar{r}{\mc{O}(-)} \ar[hookrightarrow]{d} & \on{Pic}^0_{\hyperellSource_g/\hyperell_g} \ar[hookrightarrow]{d} \\
\mc{G} \ar{r}{\mc{O}(-)} \ar{dr} & \on{Pic}_{\hyperellSource_g/\hyperell_g} \ar{d} \\
& \hyperell_g \ar[dashed, bend left = 20]{ul}{t}
\end{tikzcd}
\]
where $t$ is a section of $\mc{G} \to \hyperell_g$, for which we have the following isomorphism of group stacks over $\hyperell_g$:
\[
\begin{tikzcd}
\mc{G}^0/(2 \cdot \mc{G}^0 + \on{span}(t)) \ar{r}{\simeq} \ar{dr} & \on{Pic}^0_{\hyperellSource_g/\hyperell_g} \ar{d} \\
& \hyperell_g
\end{tikzcd}
\]
Here, $t$ is interpreted as a section of $\mc{G}^0 / (2 \cdot \mc{G}^0)$ over $\hyperell_g$ just as in step (4) of subsection~\ref{part2}.

\begin{remark} By way of example, let us explain the definition of $\mc{P}$, and prove that it is a stack. By definition, $\mc{P}(U)$ is the groupoid whose objects are pairs $(C \to U, f)$ where $C \to U$ is a hyperelliptic family over $U$ (i.e.\ an object of $\hyperell_g(U)$) and $f$ is a section of the \'etale cover $P \to U$ constructed in subsection~\ref{part2} from the family $C \to U$. A morphism $(C_1 \to U, f_1) \simeq (C_2 \to U, f_2)$ is an isomorphism $\sigma$ of families
\[
\begin{tikzcd}
C_1 \ar{rr}{\simeq} \ar{dr} && C_2 \ar{dl} \\
& U
\end{tikzcd}
\]
for which the resulting isomorphism $P_1 \simeq P_2$ of the associated spaces of Weierstrass points identifies the sections $f_1$ and $f_2$.

The descent condition is easy to check. Given an \'etale cover $V \to U$, a descent datum for $\mc{P}$ is given by a family $\wt{C} \to V$ and a section $\wt{f}$ of the resulting space of Weierstrass points, denoted $\wt{P} \to V$, along with gluing isomorphisms that take place over $V \times_U V$, which satisfy a cocycle condition on $V \times_U V \times_U V$. The cocycle condition first allows us to realize $\wt{C}$ as the pullback of a family $C \to U$, because $\hyperell_g$ is known to be a stack. By functoriality, the pullback to $V$ of the resulting space of Weierstrass points $P \to U$ is canonically identified with $\wt{P} \to V$. So effectiveness of the descent datum follows from the fact that $P \to U$ is an \'etale sheaf over $\on{Sch}_{/U}$, and, as such, satisfies a gluing axiom.
\end{remark}
In a similar way, the following points are formal consequences of subsection~\ref{part2}:
\begin{itemize}
	\item All stacks appearing in the three commutative diagrams above are algebraic.
	\item All maps to $\hyperell_g$ appearing above are representable.
	\item The isomorphism in the third diagram, which gives the desired statement about monodromy for the universal family $\hyperellSource_g \to \hyperell_g$, can be checked on pullback to schemes $U$, but this is exactly the conclusion of subsection~\ref{part2}.
\end{itemize}

To finish the proof of Theorem~\ref{monodromy-in-sym}, we need only note that any hyperelliptic family $\mc{C} \to \mc{U},$ with $\mc U$ a Deligne-Mumford stack, is pulled back from the universal family $\hyperellSource_g \to \hyperell_g$ via a map $\mc{U} \to \hyperell_g$. In this case, all constructions above can be pulled back along the same map $\mc{U} \to \hyperell_g$. Therefore, to $\mc{C} \to \mc{U}$, we can associate a stack of Weierstrass points, whose $\bz$-span maps to $\on{Pic}_{\mc{C} / \mc{U}}$, giving rise to an isomorphism analogous to that of the third commutative diagram above. This gives the desired result, by the same reasoning as in the last paragraph of subsection~\ref{part2}.

\begin{comment}
\subsection{Embedding the Symmetric Group, Take 2} \label{symbed} 

In Section~\ref{take1} we constructed the well-known embedding $S_{2g+2} \hookrightarrow \Sp_{2g}(\bz / 2 \bz)$. Beginning with 
\begin{align*} 
V &\simeq \FF_2^{2g+2} \\
t &= (1, \dots, 1) \in V \\
W &= t^{\perp}/\on{span}(t), 
\end{align*} 
we observed that the action of $S_{2g+2}$ on the basis vectors of $V$ descends to a symplectic action on $W$. Our goal in this section is to interpret $V$ and $t$ in the context of hyperelliptic curves, and thereby relate the embedding $S_{2g+2} \hookrightarrow \Sp_{2g}(\bz / 2 \bz)$ with the mod-2 Galois representation attached to a family of hyperelliptic curves. 

\subsubsection{A single hyperelliptic curve} \label{sec:single} 
Let $k$ be an algebraically closed field of characteristic zero, let $C$ be a hyperelliptic curve over $k$, and let $J$ be the Jacobian of $C$. The 
set of Weierstrass points $\{P_1, \ldots, P_{2g+2}\}$ of $C$ is uniquely determined because $g \ge 2$. With this setup, define $V$
to be the free vector space over $\FF_2$ spanned by $P_1, \ldots, P_{2g+2}$, so that 
\begin{align*} 
t &= P_1 + \cdots + P_{2g+2} \\
t^\perp &= \on{span}(P_i - P_j : i, j \in \{1, \ldots, 2g+2\}). 
\end{align*} 
The map 
\[
\begin{tikzcd}[row sep = 0.15cm] 
\on{span}_{\bz}(P_1, \ldots, P_{2g+2}) \ar{r}{\phi} & \on{Pic}_C \\
\sum_i a_i \cdot P_i  \ar[mapsto]{r} & \mc{O}_C\left( \sum_i a_i \cdot P_i \right), 
\end{tikzcd}
\]
is such that $\phi(\on{span}_{\bz}(P_i - P_j)) \subset \on{Pic}_C^0 \simeq J$. Furthermore, it can be checked that 
\begin{itemize} 
\item The resulting map $\on{span}_{\bz}(P_i - P_j) \to J$ annihilates $2\cdot (P_i - P_j)$ and $t$. Hence it descends to a map $W \defeq t^\perp / \on{span}(t) \to J[2]$ of $\FF_2$ vector spaces. 
\item This latter map is an isomorphism. 
\end{itemize} 
The second bullet point implies that the action of $\on{Aut}(C)$ on $J[2]$, \emph{a priori} contained in $\Sp_{2g}(\bz / 2 \bz)$, is in fact contained in the subgroup $S_{2g+2} \subset \Sp_{2g}(\bz / 2 \bz)$ which is determined by the vector space isomorphism $J[2] \simeq W$. For details, see \cite[Proposition 1.2.1(a)]{yelton2015thesis}. 


\subsubsection{Families of hyperelliptic curves} Let $\mc{C} \to \mc{U}$ be a family of hyperelliptic curves of genus $g$. Because all constructions in \ref{sec:single} were functorial, they can be carried out in families. Let us indicate how this is done. 
\begin{enumerate}[itemsep=0.25cm]
\item Let $\mc{P}$ be the fixed point locus of the hyperelliptic involution. Then we have a diagram 
\[
\begin{tikzcd}[column sep = 2 cm] 
	\mc{P} \ar[hookrightarrow]{r}{\text{closed emb.}} \ar[swap]{rd}{\text{\'etale}} & \mc{C} \ar{d} \\
 & \mc{U}
\end{tikzcd}
\]
For any geometric point $u \hookrightarrow \mc{U}$, the fiber $\mc{P}_u$ consists of the Weierstrass points of $\mc{C}_u$. 
\item Let $\mc{G}$ be the group stack over $\mc{U}$ which represents the sheaf associated to the following presheaf on $\on{Sch}_{/\mc{U}}$, in the \'etale topology: 
\[
T \mapsto \on{span}_\bz (\on{Hom}_{\mc{U}}(T, \mc{P})). 
\]
Representability follows by taking an \'etale cover of $\mc{U}$ which trivializes $\mc{P}$, and applying the functor $\on{span}_\bz(\bullet)$ to the gluing isomorphisms for the pullback of $\mc{P}$ to the \'etale cover. For $u \ra \mc{U}$ a geometric point, the fiber $\mc{G}_u$ equals $\on{span}_\bz(\mc{P}_u)$. 

There is a section $t \colon \mc{U} \to \mc{G}$ which is first defined on a sufficiently fine \'etale cover $\mc{U}' \to \mc{U}$ by ``adding all the Weierstrass points,'' and then descended to $\mc{U}$.  

We can also define a group substack $\mc{G}^0 \hookrightarrow \mc{G}$ via the sub-presheaf given by requiring that the coefficients of the $\bz$-linear combination sum to zero. 
\item Define a map $\Phi: \mc{G} \to \on{Pic}_{\mc{C}/\mc{U}}$ of group stacks over $\mc{U}$ as follows: given $f \in \mc{G}(T)$, we may find an \'etale cover $\sigma: T' \to T$ for which $\sigma^* f = \sum_i f_i$, for some $f_i \in \Hom_{\mc{U}}(T', \mc{P})$. Each $f_i$ gives a section of the pulled-back family $\mc{C}_{T'} \to T'$, whose image determines a relative effective Cartier divisor $D_i$. A standard descent argument shows that $\sum_i D_i$ descends to a divisor $D$ on $\mc{C}_T$, which does not depend on the chosen \'etale cover $\sigma$. We may therefore define $\Phi(f) \defeq D$. This assignment is natural in $T$, so it gives a natural transformation of functors $\mc{G} \to \on{Pic}_{\mc{C}/\mc{U}}$. The fiber $\Phi_u$ is the map $\phi$ defined in \ref{sec:single}. The map $\Phi$ restricts to a map $\mc{G}^0 \to \on{Pic}_{\mc{C}/\mc{U}}^0 \simeq \on{Jac}_{\mc{C}/\mc{U}}$. 
\item Using \ref{sec:single}, one shows that $\Phi$ induces an isomorphism 
\[
\mc{G} / (2\mc{G} + \on{span}(t)) \simeq \on{Jac}_{\mc{C}/\mc{U}}[2]
\]
of group stacks over $\mc{U}$. 
\end{enumerate} 
These constructions prove the following result: 
\begin{corollary} \label{monodromy-in-sym} 
Given a family $\mc{C} \to \mc{U}$ of hyperelliptic curves, and a geometric generic point $\ol{\eta} \hookrightarrow \mc{U}$, the monodromy group $\rho_{2}(\pi_1(U, \ol{\eta})) \subset \Sp_{2g}(\bz / 2 \bz)$ is in fact contained in $S_{2g+2} \subset \Sp_{2g}(\bz / 2 \bz)$. As a subgroup of $S_{2g+2}$, the monodromy group is given by the action of $\pi_1(U)$ on the Weierstrass points of $\mc{C}_{\ol{\eta}}$. 
\end{corollary} 
This applies to our problem as follows: 
\begin{corollary} \label{sym-W} 
For $i \in \{1, 2, 3, 4\}$, we have $\mono_{\standardFamily g i K} \subset \grouphypboth g {(i \bmod 2)} K$. 
\end{corollary} 
\begin{proof} 
We have $\on{mult}(\mono_{\standardFamily g i K}) = \chi(K)$ as subgroups of $\zh^\times$, by Remark~\ref{remark:det-rho-is-chi}. Therefore, it suffices to show that $\mono_{\standardFamily g i K}(2) \subset S_{2g+i}$. By Corollary~\ref{monodromy-in-sym}, we need only check that the monodromy action on the Weierstrass points of $\standardFamily g i K \to \standardTarget g i K$ is contained in $S_{2g+i}$. The nontrivial cases $i = 1, 3$ follow by observing that, when the defining equation is $y^2 = f(x)$ with $\deg f(x) = 2g+1$, one Weierstrass point always lies over infinity, hence is fixed under monodromy. 
\end{proof} 
\end{comment}


%
%\subsection*{Embedding the Symmetric Group, Take 2 \\ Obselete version. The new version contains much more. -James} 
%%\begin{lemma} \label{lemma:2-torsion-of-hyperelliptic-curve-action}
%%	Suppose $C/K$ is a genus-$g$ hyperelliptic curve with model $y^2 = f(x)$.
%%	If $J_C$ denotes the Jacobian of $C$,
%%	then the inclusion $\mono_{J_C}(2) \hookrightarrow \Sp_{2g}(\mathbb Z/2 \mathbb Z)$ factors through an embedding $S_{2g+2} \hookrightarrow \Sp_{2g}(\ZZ/2\ZZ)$ of the type described in Lemma~\ref{lemma:include-s}.
%%	Moreover, if $f(x) = g(x) \cdot \prod_i (x - a_i)$ is the product of a degree-$d$
%%	polynomial $g(x) \in K[x]$ with Galois group $S_d$ and a completely
%%	reducible polynomial,
%%	then
%%	$\mono_{J_C}(2) \simeq S_d \subset S_{2g+2}$, where
%%	the inclusion is taken to be the subgroup fixing a $2g+2 - d$
%%	dimensional subspace of the vector space $V$.
%%\end{lemma}
%%\begin{proof}
%%	Let $y^2 = f(x)$ be a model for a curve whose Jacobian is $J$, and let the roots of $f(x)$ be $P_1, \ldots, P_{2g+2}$.
%%	The $2$-torsion of $J$ is generated by differences of roots $P_i - P_j$ for $1 \leq i, j \leq 2g + 2$. Hence, by elementary Galois theory,
%%	if $L$ is the splitting field of $f(x)$ over $K$, we have that $\Gal(L/K) \subset S_{2g+2}$, 
%%	with the Galois group acts by permuting the roots
%%	of $f(x)$, and $S_{2g+2}$ is the full group of permutations
%%	of the roots of $f(x)$.
%%	Now, copying the notation from the proof of
%%	Lemma~\ref{lemma:include-s},
%%	if we take $V$ to be the $2g + 2$ dimensional
%%	$\mathbb F_2$ vector space generated by the $P_i$,
%%	and let $t \defeq \sum_{i=1}^{2g+2} P_i$,
%%	we see that the degree $0$ part is generated
%%	by $P_i - P_j$, (which is the subspace $t^\perp$) and satisfies the relation
%%	$t = 0$.
%%	Because the differences of the roots generate
%%	the $2$-torsion, and the resulting
%%	$\mathbb F_2$ vector space $t^\perp/(t)$ is
%%	$2g$-dimensional, we can identify $t^\perp/(t)$
%%	with
%%	$J_C[2]$.
%%	This identifies $\Gal(L/K) \subset S_{2g+2} \subset \Sp_{2g}(\mathbb Z/2\mathbb Z)$, where the first inclusion is
%%	via the action on the roots $P_i$, and the second inclusion is via the identification above.
%%
%%	In the special case that $f(x)$ is the product
%%	of a polynomial of degree $d$ with Galois group
%%	$S_d$ and a completely reducible polynomial,
%%	$\Gal(L/K)$ is the subgroup $S_d \subset S_{2g+2}$ fixing
%%	the last $2g + 2 - d$ roots, as claimed.
%%\end{proof}
%%
%%Fix an embedding $S_{2g+2} \subset \Sp_{2g}(\bz / 2 \bz)$ as described in the proof of Lemma~\ref{lemma:s2g2}. Taking $S_{n}$ to be the group of permutations of the set $\{1, \ldots, n\}$ for each positive integer $n$, we obtain an inclusion $S_{2g+1} \hookrightarrow S_{2g+2}$ by realizing $S_{2g+1}$ as the subgroup of $S_{2g+2}$ that fixes $2g+2$.
%%\todo{Add the Lemma about $S_{2g+2}$ embedding into $\Sp_{2g}(\ZZ / 2 \ZZ)$. Include a description of how this embedding can be fixed given a Galois representation.}
%
%
%In Section~\ref{take1} we described the way in which we regard $S_{2g+i}$ for $i \in \{1,2\}$ as subgroups of $\Sp_{2g}(\ZZ/2\ZZ)$. We now give a more conceptual description of this inclusion, by casting it in terms of the mod-$2$ Galois
%representation of the Jacobian of a hyperelliptic curve. In what follows, we assume $g \geq 2$.
%
%We first recall some notation from the proof of Lemma~\ref{lemma:include-s}. Let $V \simeq \FF_2^{2g+2}$, let $t = (1, \dots, 1) \in V$, and let $W = t^{\perp}/\on{span}(t)$. We saw in the proof of Lemma~\ref{lemma:include-s} that the action of $S_{2g+2}$ on the basis vectors of $V$ descends to a symplectic action on $W$. Essentially, our task is to determine what $V$ and $t$ should be in our specific setting.
%
%Let $C \ra \spec K$ be a hyperelliptic curve
%realized as the smooth projective compactification of the affine curve
%$y^2= f(x)$, with $f(x) = g(x) \cdot \prod_{i=1}^d (x -a_i)$,
%with $g(x) \in K[x]$ and $a_i \in K$ for all $i$.
%Let $J$ be the Jacobian of $C$.
%Recall that the mod-$2$ representation $\rho_{J,2} \colon \pi_1(\spec K) \to \Sp_{2g}(\ZZ/2\ZZ)$ is given by the action of $\pi_1(\spec K) = G_K$ on the $2$-torsion of the geometric fiber $J_{\overline K}$.
%Let $P_1, \ldots, P_{2g+2}$ be the Weierstrass points of $C_{\overline K}$.
%Let $V$ be the the $\mathbb F_2$ vector space 
%spanned by $P_1, \ldots, P_{2g+2}$.
%
%We claim that the $2$-torsion of $J_{\overline K}$ is given
%as an $\mathbb F_2$ vector space by the $V$-subquotient $W := \on{span}(P_i - P_j : i, j \in \{1, \dots, 2g+2\})/(\sum_{i=1}^{2g+2} P_i)$.
%To verify this is indeed the $2$-torsion,
%one can use an explicit description of the principal
%divisors on hyperelliptic curves to verify that the above subquotient
%injects into $J_{\overline K}[2]$. Since this subquotient and $J_{\overline K}[2]$ are $2g$-dimensional
%$\mathbb F_2$ vector spaces, it follows that this subquotient $W$ 
%is isomorphic to $J_{\overline K}[2]$.
%For a more detailed proof of this isomorphism, see \cite[Proposition 1.2.1(a)]{yelton2015thesis}.
%
%To conclude our description, we use the above identification of $J_{\overline K}[2]$ to recast our embedding $S_{2g+2} \ra \Sp_{2g}(\ZZ/2\ZZ)$
%in terms of the Galois group action on the Weierstrass points.
%Observe that $G_K$ permutes the roots of $f$, fixing the $a_i$ since $a_i \in K$.
%It follows that if we take $V = \on{span}(P_1, \dots, P_{2g+2})$ and $t = P_1 + \dots + P_{2g+2}$, so that $W = \on{span}(P_i - P_j : i, j \in \{1, \dots, 2g+2\})/(\sum_{i=1}^{2g+2} P_i)$, then $\mono_{J}(2) = \rho_{J, 2}(\pi_1(\spec K)) \subset S_{2g+2 - d} \hookrightarrow \Sp_{2g}(\ZZ/2\ZZ)$, where the embedding $S_{2g+2 - d} \hookrightarrow \Sp_{2g}(\ZZ/2\ZZ)$ is given by first viewing $S_{2g+2-d}$ as the subgroup of $S_{2g+2}$ fixing the $d$ Weierstrass points corresponding to $x = a_1, \ldots, a_d$, and then applying the embedding given by Lemma~\ref{lemma:include-s}.
%
%
%
\subsection{Monodromy of Hyperelliptic Families $\standardTarget g i K$ and $\hyperelliptic g K$}
\label{subsection:monodromy-of-families}

We now show that the containment in Corollary~\ref{sym-W} is an equality for the families $\standardFamily g i K \to \standardTarget g i K$.
\begin{lemma} \label{lemma:monodromy-std}
Let $g \geq 2$, and let $i \in \{1, 2, 3, 4\}$. 
\begin{enumerate}[label=(\roman*)]
	\item For any algebraically closed field $L$ which is a subfield of $\mathbb C$, we have $\mono_{\standardFamily g i L}= \hyphat_{2g+2-(i \bmod 2)}$. 
	\item For any number field $K$, we have $\mono_{\standardFamily g i K}= \grouphypboth g {(i \bmod 2)} {K}$. 
\end{enumerate} 
\end{lemma}
\begin{proof}
(i) $\Leftrightarrow$ (ii): we have a map of short exact sequences 
\begin{equation}
	\label{equation:}
	\begin{tikzcd}
		0 \ar {r}  &  \mono_{\standardFamily g i {\overline K}} \ar {r}\ar{d} & \mono_{\standardFamily g i K} \ar {r}{\mult}\ar{d} & \chi(K) \ar {r}\ar[equal]{d} & 0 \\
			0 \ar {r} &  \hyphat_{2g+2-(i \bmod 2)} \ar {r} & \grouphypboth g {(i \bmod 2)} K \ar[swap]{r}{\mult} & \chi(K) \ar {r} & 0.
	\end{tikzcd}\end{equation}
By the Five Lemma, the second vertical map is an isomorphism if and only if the first is.

Proof of (i): By Corollary~\ref{sym-W}, and because $L$ is a subfield of $\bc$, we have containments
\[
H_{\standardFamily g i \bc} \subset H_{\standardFamily g i L} \subset \hyphat_{2g+2-(i \bmod 2)}. 
\]
Therefore, it suffices to show that $H_{\standardFamily g i \bc} = \hyphat_{2g+2-(i \bmod 2)}$. For $i \in \{1, 2\}$, this follows from~\cite[Th\'eor\`eme 1]{acampo:tresses-monodromie-et-le-groupe-symplectique}, since the \'etale fundamental group is the profinite completion of the topological fundamental group. To complete the proof we need only show that, when $i \in \{3, 4\}$, we have $\mono_{\standardFamily g i {\mathbb C}} = \mono_{\standardFamily g {i-2} {\mathbb C}}$. 

For this, it suffices to construct a deformation retract 
\[
	\phi: \standardTarget g {i-2} {\mathbb C} \times [0, 1] \to \standardTarget g {i-2} {\mathbb C}
\]
of $\standardTarget g {i-2} {\mathbb C}$ onto $\standardTarget g i {\mathbb C}$, which is done as follows. Let $n \defeq 2g+2-(i \bmod 2)$. Then $\standardTarget g {i-2} {\mathbb C}$ parameterizes unordered $n$-tuples of distinct points in $\ba^1_\bc$, and $\standardTarget g i {\mathbb C}$ parameterizes those which sum to zero. At time $t \in [0, 1]$, we define 
\[
\phi_t \colon \{z_i\}_{i=1}^n \mapsto \left\{z_i - t \cdot \frac{z_1 + \cdots + z_n}{n}\right\}_{i=1}^n, 
\]
where the $n$-tuple on the right sums to zero by construction. This $\phi$ is continuous, as desired. In fact, $\phi$ is regular: its coordinate functions are obtained by expressing the elementary symmetric polynomials in the right hand side $n$-tuple as polynomials in (the elementary symmetric polynomials of the $z_i$) and $t$. 
\end{proof}

\begin{corollary}
	\label{monodromy-stack} 
	Let $g \geq 2$. We have that $\mono_{\hyperellipticSource g K} = \grouphyp g {K}$,
	where $\hyperellipticSource g K \ra \hyperelliptic g K$ is the universal family over the moduli stack of hyperelliptic curves.
\end{corollary}
\begin{proof}
\emph{A priori}, we have the containments 
\[
\mono_{\standardFamily g 2 K} \subset \mono_{\hyperellipticSource g K} \subset \grouphyp g K, 
\]
the latter following from Corollary~\ref{monodromy-in-sym}. But Lemma~\ref{lemma:monodromy-std} implies that $\mono_{\standardFamily g 2 K} = \grouphyp g K$. 
\end{proof}


\begin{comment} 
\subsection{Monodromy of Hyperelliptic Families $\standardTarget g i K$ and $\hyperelliptic g K$}
% - Version with irreducibility of generic fiber (either cut this or the previous two subsections
\label{subsection:monodromy-of-families}

In this section, we compute the monodromy of $\standardTarget g i K$ and $\hyperelliptic g K$. To start, we find
the monodromy associated to $\hyperelliptic g K$, by relating it to that of $\standardTarget g 2 K$, which was
previously computed in~\cite[Th\'eor\`eme 1]{acampo:tresses-monodromie-et-le-groupe-symplectique}.

\begin{lemma}
	\label{lemma:monodromy-stack}
	For $L$ an algebraically closed field of characteristic $0$, we have $\mono_{\hyperellipticSource g L} = \hyphat_{2g+(i \bmod 2)}$.
\end{lemma}
\begin{proof}
First, note that $\mono_{\standardFamily g 2 \bc} = \hyphat_{2g+2}$, as shown in~\cite[Th\'eor\`eme 1]{acampo:tresses-monodromie-et-le-groupe-symplectique}, since the \'etale fundamental group is the profinite completion of the topological fundamental group. 
Invariance of the fundamental group under base field extension between algebraically closed fields of characteristic $0$
for quasi-projective normal schemes
(see \cite[Expos\'e XIII, Proposition 4.6]{noopsortSGA1Grothendieck1971}, taking $Y = \spec L$ there)
%\todo{this reference is overkill and we dont need it, but it seemed easier to use it, should we remove it or is it ok?}
guarantees the same for any algebraically closed field $L$ of characteristic $0$.
%Since we have a natural map
%$\standardFamily g 2 L \rightarrow \hyperelliptic g L$,
%it follows that
%$\mono_{\hyperellipticSource g L} \subset \hyphat_{2g+(i \bmod 2)}$.
%To show equality, it suffices to verify that the $\bmod 2$ monodromy
%of $\hyperelliptic_{g,L}$ is precisely $S_{2g+2} \subset \Sp_{2g}(\bz/2\bz)$.

Next, $\hyperelliptic g L$ can be realized as a $\bz/2\bz$-gerbe over
the stack theoretic quotient $[V/\PGL_2]$ with $V \subset \mathbb P^{2g+2}$ the open subscheme
parameterizing $(2g+2)$-tuples of distinct points in $\mathbb P^1$.
From the commutative diagram
\begin{equation}
	\label{equation:}
	\begin{tikzcd} 
		\standardTarget g 2 L \ar {r} \ar {d} & \hyperelliptic g L \ar {d} \\
		V \ar {r} & \left[ V/\PGL_2 \right],
	\end{tikzcd}\end{equation}
the composite map
$\pi_1(\standardTarget g 2 L) \ra \pi_1(V) \ra \pi_1([V/\PGL_2])$
is surjective.
Indeed, this holds because 
$\standardTarget g 2 L \subset V$ is a dense open in a normal scheme
and $V \ra \left[ V/\PGL_2 \right]$ is surjective with irreducible geometric
generic fiber (see, for example,~\cite[Proposition 5.2]{landesman-swaminathan-tao-xu:rational-families}). These maps give a commutative diagram
\[
\begin{tikzcd}
& \pi_1(\standardTarget{g}{2}{L}) \ar[ld, dashed] \ar[d, twoheadrightarrow] \\
\pi_1(\hyperelliptic{g}{L}) \ar[r] & \pi_1(\left[ V/\PGL_2 \right])
\end{tikzcd}
\]
On the other hand, since $\hyperelliptic g L$ is a $\bz/2$-gerbe over $\left[ V/\PGL_2 \right]$, we have the following fibration exact sequence
\[
\begin{tikzcd}
\bz/2\bz \simeq \pi_1([\on{pt} / (\bz/2\bz)]) \ar[r] & \pi_1(\hyperelliptic{g}{L}) \ar[r] & \pi_1(\left[ V/\PGL_2 \right]) \ar[r] & \pi_0([\on{pt} / (\bz / 2 \bz)]) \simeq \{0\}. 
\end{tikzcd}
\]
It follows that $\pi_1(\hyperelliptic{g}{L})$ is spanned by the images of $\bz/2\bz$ (via the hyperelliptic involution) and $\pi_1(\standardTarget{g}{2}{L})$ (via the dashed map above, corresponding to the family $\standardFamily{g}{2}{L} \to \standardTarget{g}{2}{L}$). Now, since the hyperelliptic involution acts on $\GSp_{2g}(\widehat{\bz})$ by $-1$, which is already contained in $\hyphat_{2g+2}$, it follows that $\hyphat_{2g+2} = \mono_{\standardFamily g 2 L} = \mono_{\hyperellipticSource g L}$,
as desired.
\end{proof}

Using a topological deformation retract, we compute the monodromy of $\standardTarget g i K$ from that of $\standardTarget g 2 K$.
\begin{lemma} \label{lemma:monodromy-std}
Let $g \geq 2$, and let $i \in \{1, 2, 3, 4\}$. 
For any algebraically closed field $L$ of characteristic $0$, we have $\mono_{\standardFamily g i L}= \hyphat_{2g+(i \bmod 2)}$. 
\end{lemma}
\begin{proof}
Since the fundamental group over $L$ is isomorphic to that over $\mathbb C$ by \cite[Expos\'e XIII, Proposition 4.6]{noopsortSGA1Grothendieck1971},
it suffices to prove the result over $\mathbb C$.
For $i \in \{1, 2\}$, this follows over $\mathbb C$ 
from~\cite[Th\'eor\`eme 1]{acampo:tresses-monodromie-et-le-groupe-symplectique}, since the \'etale fundamental group is the profinite completion of the topological fundamental group.

To complete the proof we need only show that, when $i \in \{3, 4\}$, we have $\mono_{\standardFamily g i {\mathbb C}} = \mono_{\standardFamily g {i-2} {\mathbb C}}$. 
For this, it suffices to construct a deformation retract 
\[
	\phi: \standardTarget g {i-2} {\mathbb C} \times [0, 1] \to \standardTarget g {i-2} {\mathbb C}
\]
of $\standardTarget g {i-2} {\mathbb C}$ onto $\standardTarget g i {\mathbb C}$, which is done as follows. Let $n \defeq 2g+(i \bmod 2)$. Then $\standardTarget g {i-2} {\mathbb C}$ parameterizes unordered $n$-tuples of distinct points in $\ba^1_\bc$, and $\standardTarget g i {\mathbb C}$ parameterizes those which sum to zero. At time $t \in [0, 1]$, we define 
\[
\phi_t \colon \{z_i\}_{i=1}^n \mapsto \left\{z_i - t \cdot \frac{z_1 + \cdots + z_n}{n}\right\}_{i=1}^n, 
\]
where the $n$-tuple on the right sums to zero by construction. This $\phi$ is continuous and its coordinate functions are obtained by expressing the elementary symmetric polynomials in the right hand side $n$-tuple as polynomials in $t$ and the elementary symmetric polynomials of the $z_i$.
\end{proof}

As the final result of this section, we deduce the monodromy of the hyperelliptic families over a number field from the monodromy over $\overline {\mathbb Q}$.
\begin{corollary}
	\label{corollary:monodromy-over-number-field} 
	Let $g \geq 2$. For $K$ a number field, we have $\mono_{\hyperellipticSource g K} = \grouphyp g K$ and
$\mono_{\standardFamily g i K}= \grouphypboth g {(i \bmod 2)} {K}$.
\end{corollary}
\begin{proof}
We have a map of short exact sequences 
\begin{equation}
	\label{equation:}
	\begin{tikzcd}
		0 \ar {r}  &  \mono_{\standardFamily g i {\overline K}} \ar {r}\ar{d} & \mono_{\standardFamily g i K} \ar {r}{\mult}\ar{d} & \chi(K) \ar {r}\ar[equal]{d} & 0 \\
			0 \ar {r} &  \hyphat_{2g+(i \bmod 2)} \ar {r} & \grouphypboth g {(i \bmod 2)} K \ar[swap]{r}{\mult} & \chi(K) \ar {r} & 0.
	\end{tikzcd}\end{equation}
By the Five Lemma, the second vertical map is an isomorphism if and only if the first one is,
so the result for $\hyperelliptic g K$ follows from Lemma~\ref{lemma:monodromy-stack}
and the result for $\standardFamily g i K$ follows from Lemma~\ref{lemma:monodromy-std}
\end{proof}


\end{comment} 

\subsection{Finishing the Proof}\label{32isnotagoodscoreline}

We are now in position to complete the proof of Theorem~\ref{mainbldg}.
The main input to the proof is the following general theorem from \cite{landesman-swaminathan-tao-xu:rational-families}.
\begin{theorem}[\protect{\cite[Theorem 1.1]{landesman-swaminathan-tao-xu:rational-families}}]

	\label{theorem:main}
	Let $B, n > 0$, and suppose that the rational family $A \to U$ is non-isotrivial and has big monodromy, meaning that $\mono_A$ is open in $\GSp_{2g}(\zh)$. Let $\delta_\QQ$ be the index of the closure of the commutator subgroup of $\mono_A$ in $\mono_A \cap \Sp_{2g}(\zh)$, and let $\delta_K = 1$ for $K \neq \QQ$. Then $[\mono_A : \mono_{A_u}] \geq \delta_K$ for all $u \in U(K)$, and we have the following asymptotic statements:
			\[
				\frac{|\{u \in U(K) \cap \mathcal{O}_K^r : \lVert u \rVert \le B,\, [\mono_A : \mono_{A_u}] = \delta_K\}|}{|\{u \in U(K) \cap \mc{O}_K^r : \lVert u \rVert \le B\}|} = 1 + O((\log B)^{-n}), \text{ and}
			\]
\[
				\frac{|\{u \in U(K) : \on{Ht}(u) \leq B,\, [\mono_A : \mono_{A_u}] = \delta_K\}|}{|\{u \in U(K) : \on{Ht}(u) \le B\}|} = 1 + O((\log B)^{-n}),
			\]
	 where the implied constants depend only on $A \to U$ and $n$.
\end{theorem} 
\begin{proof}[Proof of Theorem~\ref{mainbldg}]
	First, we explain how Theorem~\ref{theorem:main} applies to the standard families $\standardFamily g i K \rightarrow \standardTarget g i K$. By Lemma~\ref{lemma:monodromy-std}(ii), these families have big monodromy, so Theorem~\ref{theorem:main} applies.  Lemma~\ref{lemma:monodromy-std}(ii) says that $H_{\standardFamily g i K} = \grouphyp g K$. With this in mind, Corollary~\ref{theorem:r=2:special} implies that $\delta_\bq = 2$ in the statement of Theorem~\ref{theorem:main}. 

	%Let $\tau_g: \mg \ra \ag$ denote the Torelli map. By abuse of notation
	%if $F$ is a family of curves, we also let $\tau_g(F)$ denote the corresponding family of abelian varieties.
	%Observe that
	%since 
	%$\standardTarget g i K$ has big geometric monodromy
	%by Lemma~\ref{lemma:geometric-monodromy-of-standard-family},
	%the statement for the four families $\standardTarget g i K$
	%follows from Theorem~\ref{theorem:main}
	%once we know that $\delta_\bq = 2$, with $\delta_\bq$ as in Theorem~\ref{theorem:main}.
	%Indeed,
	%\begin{align*}
	%	\delta_\bq &=\left[\mono_{\tau_g(\standardFamily g i K)} \cap \Sp_{2g}(\wh{\ZZ}) : \left[\mono_{\tau_g(\standardFamily g i K)}, \mono_{\tau_g(\standardFamily g i K)}\right]\right] \\
%&=\left[\mono_{\standardFamily g i K} \cap \Sp_{2g}(\wh{\ZZ}) : \left[\mono_{\standardFamily g i K},\mono_{\standardFamily g i K}\right]\right] \\
%&= \left[\hyphat_{2g+(i\bmod 2)}:\left[\grouphypboth{g}{(i\bmod 2)}{K}, \grouphypboth{g}{(i \bmod 2)}{K}\right]\right] & \text{(by Lemma~\ref{lemma:geometric-monodromy-of-standard-family})}\\
%&= 2. & \text{(by Corollary~\ref{theorem:r=2:special})}
%	\end{align*}
%Here, we are using that $\mono_{\tau_g(\standardFamily g i K)} = \mono_{\standardFamily g i K}$,
%since the monodromy of a locus in $\mg$ agrees with the monodromy of its image in $\ag$ under $\tau_g$, as both can be identified with the monodromy action on
%the first cohomology group.

Next, we apply Theorem~\ref{theorem:main} to a rational family $C \rightarrow U$ represented by a map $U \to \hyperelliptic g K$ with connected geometric generic fiber. This hypothesis implies that $\pi_1(U) \to \pi_1(\hyperelliptic g K)$ is a surjection, cf.\ \cite[Corollary 5.3]{landesman-swaminathan-tao-xu:rational-families}, so the monodromy group of $C \to U$ is equal to that of the universal family over $\hyperelliptic g K$, and Corollary~\ref{monodromy-stack} implies that the universal family over $\hyperelliptic g K$ has monodromy group $\grouphyp g K$. At this point, Corollary~\ref{theorem:r=2:special} implies $\delta_\bq = 2$, as before. 
\begin{comment}
\textcolor[rgb]{1,0,0}{To conclude, we only need deal with the case $g = 1$.
We separate this case mostly for notational reasons, as $\Sp_2(\mathbb Z/2 \mathbb Z) \simeq S_3$,
and so there is no copy of $S_{2 \cdot 1 + 2}$ in $\Sp_2(\mathbb Z/2 \mathbb Z)$.
The proof is then essentially the same as the $g \geq 2$ case, where we adjust
two of the inputs:
First, ~\cite[Th\'eor\`eme 1]{acampo:tresses-monodromie-et-le-groupe-symplectique} states that in the case $g= 1$, the geometric monodromy of $\standardFamily 1 2 K$ is $S_3$ instead using 
Theorem~\ref{theorem:main} as we did in the $g \geq 2$ case, we use
\cite[Theorem 1.14]{zywina2010hilbert} for the case $K \neq \mathbb Q$
and \cite[Theorem 1.15]{zywina2010hilbert} for the case $K = \mathbb Q$,
which is analogous to Theorem~\ref{theorem:main} in the case $g=1$, but
yields better estimates for the error term.}
\end{comment}
\end{proof}
