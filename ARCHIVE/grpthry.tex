\section{Reduction to the Mod-$\ell$ Case}
	
\todo{Notation for the symmetric group.}

\todo{Mention Serre groups}

\todo{Mention mod 2 $f(x)$ has Galois group $S_n$ is maximal}


    Let $C$ be a smooth projective hyperelliptic curve of genus $g$ with Jacobian $J_C$. In this section, we reduce the global problem of determining whether the adelic representation $\rho_{J_C}$ has largest possible image to the local problem of checking whether the mod-$\ell$ representations $\rho_{J_C,\ell}$ have largest possible image.

    We first claim that the datum of $\rho_{J_C}(G_K) \subset \GSp_{2g}(\wh{\ZZ})$ is equivalent to the datum of $\rho_{J_C}(G_K) \cap \Sp_{2g}(\wh{\ZZ}) \subset \Sp_{2g}(\wh{\ZZ})$. To prove this claim, consider the short exact sequence

    \begin{center}
    \begin{tikzcd}
    1 \arrow{r} & \rho_{J_C}(G_K) \cap \Sp_{2g}(\wh{\ZZ}) \arrow{r} & \rho_{J_C}(G_K) \arrow{r}{\on{mult}} & \chi(G_K) \arrow{r} & 1
    \end{tikzcd}
    \end{center}

    If     $\rho_{J_C}(G_K)$ \todo{???}

    We next specify exactly what it means to say that $\rho_{J_C}$ has largest possible image. Let $f \in K[x]$ be a polynomial of degree $2g+2$ with the property that $C$ is a smooth projective model of the affine plane curve defined by the Weierstrass equation $y^2 = f(x)$, and let the necessarily distinct roots of $f$ in $\ol{K}$ be denoted by $p_1, \dots, p_{2g+2}$. Recall by the definition of the Jacobian that we have a bijection $J_C(\ol{K}) \to \on{Pic}^0 C_{\ol{K}}$. The image of the $2$-torsion subgroup $J_C[2](\ol{K})$ under this bijection
    \todo{Explain how $S_{2g+2}$ embeds into $\GSp_{2g}$.}
    We say that the image of $\rho_{J_C}$ is maximal if it is equal to the preimage under the projection map $\GSp_{2g}(\wh{\ZZ}) \twoheadrightarrow$\todo{to finish}

